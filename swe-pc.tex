\documentclass{article}
\usepackage[utf8]{inputenc}
\usepackage[colorlinks,linkcolor=blue,citecolor=blue,urlcolor=blue]{hyperref}
\usepackage[round]{natbib}
\usepackage{doi}
\usepackage{amsmath}
\usepackage{amssymb}
\usepackage{mathtools}
\usepackage{xcolor}

\title{Stochastic shallow water formulations}
\author{James Shaw}

\begin{filecontents}{swe-pc.bib}
@article{delis2000,
	title={Evaluation of some approximate {Riemann} solvers for transient open channel flows},
	author={Delis, AT and Skeels, CP and Ryrie, SC},
	journal={Journal of Hydraulic Research},
	volume={38},
	number={3},
	pages={217--231},
	year={2000},
	doi={10.1080/00221680009498339},
	publisher={Taylor \& Francis Group}
}

@article{ge2008,
	title={Stochastic solution for uncertainty propagation in nonlinear shallow-water equations},
	author={Ge, Liang and Cheung, Kwok Fai and Kobayashi, Marcelo H},
	journal={Journal of Hydraulic Engineering},
	volume={134},
	number={12},
	pages={1732--1743},
	year={2008},
	doi={10.1061/(ASCE)0733-9429(2008)134:12(1732)},
	publisher={American Society of Civil Engineers}
}

@phdthesis{ge2009,
	title={Polynomial Chaos Representation of Uncertainties in Nonlinear Shallow-Water Equations for Flood Hazard Assessment},
	author={Liang Ge},
	year={2009},
	school={University of Hawai`i at Manoa}
}

@inproceedings{hosder2007,
	title={Efficient sampling for non-intrusive polynomial chaos applications with multiple uncertain input variables},
	author={Hosder, Serhat and Walters, Robert and Balch, Michael},
	booktitle={48th AIAA/ASME/ASCE/AHS/ASC Structures, Structural Dynamics, and Materials Conference},
	pages={1939},
	year={2007},
	doi={10.2514/6.2007-1939}
}

@article{kesserwani2008,
	title={Riemann solvers with {Runge--Kutta} discontinuous {Galerkin} schemes for the {1D} shallow water equations},
	author={Kesserwani, G and Ghostine, R and Vazquez, J and Ghenaim, Abdellah and Mos{\'e}, Robert},
	journal={Journal of Hydraulic Engineering},
	volume={134},
	number={2},
	pages={243--255},
	year={2008},
	doi={10.1061/(ASCE)0733-9429(2008)134:2(243)},
	publisher={American Society of Civil Engineers}
}
\end{filecontents}

\newcommand{\dee}{\mathrm{d}}
\newcommand{\diff}{\:\mathrm{d}}
\newcommand{\eigenval}{\tilde{a}}
\newcommand{\eigenvect}{\tilde{\vect{e}}}
\newcommand{\Ensemble}[1]{\left\langle #1 \right\rangle}
\newcommand{\Mag}[1]{\left\lvert #1 \right\rvert}
\newcommand{\randomroot}{\hat{\xi}}
\newcommand{\riemannflux}{\widetilde{\vect{F}}}
\newcommand{\TODO}[1]{\textcolor{purple}{TODO: \emph{#1}}}
\newcommand{\vect}{\mathbf}
\newcommand{\zbar}{\bar{z}}
\newcommand{\zstar}{z^\star}

\begin{document}

\maketitle

\subsection*{Deterministic FV1 SWE solver}

\begin{align}
	\frac{\partial \vect{U}}{\partial t} + \frac{\partial \vect{F}}{\partial x} = \vect{S} \label{eqn:swe}
%
\intertext{where}
%
	\vect{U} = \left[ h, q \right]^\intercal \\
	\vect{F} = \left[ q,  \frac{q^2}{h} + \frac{gh^2}{2} \right]^\intercal \\
	\vect{S} = \left[ 0, -gh \frac{\partial z}{\partial x} \right]^\intercal
\end{align}
The first-order finite volume discretisation is
\begin{align}
	\vect{U}_i^{(n+1)} = \vect{U}_i^{(n)} - \frac{\Delta t}{\Delta x}
	\left( \riemannflux_{i+1/2}^{(n)} - \riemannflux_{i-1/2}^{(n)}
	- \Delta x \vect{S}_i^{(n)} \right) \label{eqn:fvswe} \\
	\vect{S}_i^{(n)} = \left[ 0, -g h_i^{(n)} \frac{z_{i+1/2} - z_{i-1/2}}{\Delta x} \right]^\intercal \label{eqn:sourceterm}
\end{align}
The numerical flux $\riemannflux_{i-1/2}$ is \citep{kesserwani2008}
\begin{align}
	\riemannflux_{i-1/2} &= \frac{1}{2} \left( \vect{F}_{i-1} + \vect{F}_i 
	- \sum_{j=1}^2 \alpha_{i-1/2}^j \: \psi(\eigenval_{i-1/2}^j) \: \eigenvect_{i-1/2}^j \right)
%
\intertext{where the eigenvalues $\eigenval$ and eigenvectors $\eigenvect$ are \citep{delis2000}}
%
	\eigenval_{i-1/2}^{1,2} &= \tilde{u}_{i-1/2} \pm \tilde{c}_{i-1/2} \\
	\eigenvect_{i-1/2}^{1,2} &= \left[ 1, \eigenval_{i-1/2}^{1,2} \right]^\intercal \\
	\tilde{u}_{i-1/2} &= \frac{u_{i-1} \sqrt{g h_{i-1}} + u_i \sqrt{g h_i}}{\sqrt{g h_{i-1}} + \sqrt{g h_i}} 
	= \frac{u_{i-1} \sqrt{h_{i-1}} + u_i \sqrt{h_i}}{\sqrt{h_{i-1}} + \sqrt{h_i}} \\
	\tilde{c}_{i-1/2} &= \sqrt{g \tilde{h}_{i-1/2}} \\
	\tilde{h}_{i-1/2} &= \frac{h_{i-1} + h_i}{2}
\end{align}
The average wave strength $\alpha$ is calculated explicitly \citep{delis2000}
\begin{align}
	\alpha_{i-1/2}^{1,2} = \frac{\Delta q_{i-1/2} + \left( - \tilde{u}_{i-1/2} \pm \tilde{c}_{i-1/2} \right) \Delta h_{i-1/2}}{\pm 2 \tilde{c}_{i-1/2}}
\end{align}
The entropy correction function $\psi(\eigenval_{i-1/2}^j)$ is
\begin{align}
	\psi(\eigenval_{i-1/2}^j) &=
	\begin{cases}
		\Mag{\eigenval_{i-1/2}^j}
			& \text{if $\Mag{\eigenval_{i-1/2}^j} \geq \delta_{i-1/2}^j$,} \\
		\left[ \left(\eigenval_{i-1/2}^j\right)^2 +
		\left(\delta_{i-1/2}^j\right)^2 \right] / 2 \delta_{i-1/2}^j
			& \text{otherwise}
	\end{cases}
%
\intertext{where $\vect{\delta}_{i-1/2}$ is \TODO{using Georges' formulation or Delis?}}
%
	\vect{\delta} &= \left[ \max(0, ) \right]^\intercal
\end{align}

\subsection*{Stochastic formulation with intrusive polynomial chaos}

Following \citet{ge2008}, we derive the stochastic shallow water equations using intrusive polynomial chaos with Gaussian uncertainties.
We introduce a random variable $\xi$ and express the flow vector as a polynomial in random space,
\begin{align}
	\vect{U}(x, t, \xi) = \sum_{p=0}^P \vect{U}_p(x,t) \Phi_p(\xi)
\end{align}
where $\vect{\Phi} = \left[ \Phi_0, \ldots, \Phi_P \right]$ is the probabilists' Hermite polynomial basis that represents the flow uncertainty with a Gaussian distribution.
Equation~\eqref{eqn:swe} then becomes
\begin{align}
	\frac{\partial}{\partial t} \sum_{p=0}^P \vect{U}_j \Phi_p + \frac{\partial \vect{F}(\vect{U}(x, t, \xi))}{\partial x} = \vect{S}(\vect{U}(x, t, \xi))
%
\intertext{and equation~\eqref{eqn:fvswe} becomes}
%
	\sum_{p=0}^P \vect{U}_{i,p}^{(n+1)} \Phi_p = \sum_{p=0}^P \vect{U}_{i,p}^{(n)} \Phi_p - \frac{\Delta t}{\Delta x}
	\left( \riemannflux_{i+1/2}^{(n)} - \riemannflux_{i-1/2}^{(n)}
	- \Delta x \vect{S}_i^{(n)} \right) \label{eqn:randomswe}
\end{align}
The stochastic source term $\vect{S}$ involves stochastic flow and stochastic topography and so equation~\eqref{eqn:sourceterm} becomes
\begin{align}
	\vect{S}_i^{(n)} = \left[ 0, -g \sum_{p=0}^P h_{i,p}^{(n)} \Phi_p \frac{1}{\Delta x} \sum_{p'=0}^P \left( z_{i+1/2, p'} - z_{i-1/2, p'} \right) \Phi_{p'} \right]^\intercal \label{eqn:randomsourceterm}
\end{align}

Next we use the ensemble average defined as
\begin{align}
	\Ensemble{\Phi_i(\xi) \Phi_j(\xi)} = \int_{-\infty}^\infty \Phi_i(\xi) \Phi_j(\xi) W(\xi) \dee \xi
%
\intertext{where the weight function $W(\xi)$ for the Hermite polynomial basis is}
%
	W(\xi) = \frac{1}{\sqrt{2\pi}} \exp(-\xi^2/2)
\end{align}
Since the Hermite basis is orthogonal then
\begin{align}
	\Ensemble{\Phi_i \Phi_j} = \Ensemble{\Phi_i^2} \delta_{ij}
\end{align}
Using these properties of the polynomial chaos basis, we project equation~\eqref{eqn:randomswe} onto the bases $\Phi_l, l = 0, \ldots, P$ to obtain $P+1$ decoupled equations,
\begin{align}
	\vect{U}_{i,l}^{(n+1)} = \vect{U}_{i,l}^{(n)} - \frac{\Delta t}{\Delta x \Ensemble{\Phi_l^2}}
	\left( \Ensemble{ \riemannflux_{i+1/2}^{(n)} \Phi_l} - \Ensemble{\riemannflux_{i-1/2}^{(n)} \Phi_l}
	- \Delta x \Ensemble{ \vect{S}_i^{(n)} \Phi_l} \right) \label{eqn:randomprojectedswe}
%
\intertext{where the ensemble average of the source term is obtained from equation~\eqref{eqn:randomsourceterm}}
%
	\Ensemble{ \vect{S}_i^{(n)}, \Phi_l} = \left[ 0, - \frac{g}{\Delta x} \sum_{p=0}^P \sum_{p'=0}^P h_{i,p}^{(n)} \left( z_{i+1/2, p'} - z_{i-1/2, p'} \right) \Ensemble{\Phi_p \Phi_{p'} \Phi_l} \right]^\intercal
\end{align}
The flux term in equation~\eqref{eqn:randomprojectedswe} is approximated by Gauss-Hermite quadrature, 
\begin{align}
	\Ensemble{\riemannflux_{i-1/2}(\vect{U}(x, t, \xi)) \: \Phi_l(\xi)} &=
	\int_{- \infty}^\infty \riemannflux_{i-1/2}
	\Phi_l W(\xi) \diff \xi \\
	&\approx \sum_{j=1}^J w_j \riemannflux_{i-1/2}(\vect{U}(x, t, \xi_j))
	\Phi_l(\xi_j) W(\xi_j)
\end{align}
where $w_j$ are the quadrature weights and $\xi_j$ are the quadrature points in random space.

\subsection*{Calculating probability density functions}
For a given flow variable $u = h$ or $u = q$, the probability density function $f(u)$ can be calculated for a given element $i$ and time $n$,
\begin{subequations}
\begin{align}
	f(u) = \sum_{i=1}^n \Mag{ \sum_{p=0}^P u_{i,p}^{(n)} \frac{\dee \Phi_p}{\dee \xi}(\randomroot_i)}^{-1} W(\randomroot_i)
%
	\intertext{where $\randomroot_i$, $i=1, \ldots, n$ are the real roots of the polynomial}
%
	u - \sum_{p=0}^P u_{i,p}^{(n)} \Phi_p(\xi) = 0
\end{align}
\end{subequations}
which can be calculated numerically for a given value of $u$.

\subsection*{Well-balancing}
There are two ingredients for well-balancing: a modified source term discretisation, and modified inputs to the Riemann solver.
\begin{align}
	z_{i-1/2} &= \frac{z_{i-1} + z_i}{2} \\
	\Delta z_i &= z_{i+1/2} - z_{i-1/2} \\
	\eta_i &= h_i + z_i \\
	h_{i-1/2,L} &= \eta_{i-1} - z_{i-1/2} \\
	h_{i-1/2,R} &= \eta_{i} - z_{i-1/2} \\
	\bar{h}_i &= \frac{h_{i-1/2,R} + h_{i+1/2,L}}{2} \\
\end{align}
The deterministic source term $\vect{S}_i$ is
\begin{align}
	\vect{S}_i &= \left[ 0, -g \bar{h}_i \frac{\Delta z_i}{\Delta x} \right]^\intercal
%
\intertext{which can be written as}
%
	\vect{S}_i &= \left[ 0, -\frac{g}{\Delta x} \left\{
	\left(\frac{h_i}{2} + \frac{z_i}{4} \right) \left( z_{i+1} - z_{i-1}\right)
	- \frac{1}{8} \left( z_{i+1}^2 - z_{i-1}^2 \right)
	\right\} \right]^\intercal
\end{align}
The stochastic source term is then
\begin{align}
	\begin{split}
	\vect{S}_i &= \left[0, -\frac{g}{\Delta x} \left\{
	\sum_{p=0}^P \sum_{p'=0}^P
	\left(\frac{h_{i,p}}{2} + \frac{z_{i,p}}{4} \right) \left( z_{i+1,p'} - z_{i-1,p'}\right) \Phi_p \Phi_{p'} \right.\right.\\
	&\left.\left. - \frac{1}{8} \left(
	\sum_{p=0}^P \sum_{p'=0}^P
	\left( z_{i+1,p}\:z_{i+1,p'} - z_{i-1,p}\:z_{i-1,p'} \right)
	\Phi_p \Phi_{p'} \right)
	\right\} \right]^\intercal
	\end{split}
%
\intertext{and the ensemble average is}
%
	\begin{split}
	\Ensemble{\vect{S}_i, \Phi_l} &=
	\left[0, -\frac{g}{\Delta x} \left\{
	\sum_{p=0}^P \sum_{p'=0}^P
	\left(\frac{h_{i,p}}{2} + \frac{z_{i,p}}{4} \right) \left( z_{i+1,p'} - z_{i-1,p'}\right) \Ensemble{\Phi_p \Phi_{p'} \Phi_l} \right.\right.\\
	&\left.\left. - \frac{1}{8} \left(
	\sum_{p=0}^P \sum_{p'=0}^P
	\left( z_{i+1,p}\:z_{i+1,p'} - z_{i-1,p}\:z_{i-1,p'} \right)
	\Ensemble{\Phi_p \Phi_{p'} \Phi_l} \right)
	\right\} \right]^\intercal
	\end{split}
\end{align}
The Riemann solver inputs are $\left[h_{i-1/2,L}, q_{i-1}\right]^\intercal$, $\left[h_{i-1/2,R}, q_i\right]^\intercal$.

\subsection*{Nonintrusive formulation}

The nonintrusive formulation samples the deterministic model with a set of $N$ initial conditions.
The initial conditions $\vect{U}^{(0)}$ are calculated using Gauss-Hermite quadrature points $\xi_j$ such that
\begin{align}
	\vect{U}^{(0)}(x, \xi_j) = \sum_{p=0}^P \vect{U}_p^{(0)}(x) \Phi_p(\xi_j)
\end{align}
The deterministic solver calculates the solutions $\vect{U}^{(n)}(x, \xi_1), \ldots \vect{U}^{(n)}(x, \xi_N)$.
Now we can write a matrix equation
\begin{align}
	\begin{bmatrix}
		\Phi_0(\xi_0) & \cdots & \Phi_P(\xi_0) \\
		\vdots & \ddots & \vdots \\
		\Phi_0(\xi_0) & \cdots & \Phi_P(\xi_N)
	\end{bmatrix}
	\begin{bmatrix}
		\vect{U}_0^{(n)}(x) \\
		\vdots \\
		\vect{U}_P^{(n)}(x)
	\end{bmatrix}
	=
	\begin{bmatrix}
		\vect{U}^{(n)}(x, \xi_1) \\
		\vdots \\
		\vect{U}^{(n)}(x, \xi_N)
	\end{bmatrix}
\end{align}
For better convergence, the matrix equation can be over-constrained with $N > P$, and the unknown coefficients $\vect{U}_0^{(n)}(x), \ldots, \vect{U}_P^{(n)}(x)$ are found using a least-squares approach \citep{hosder2007}.

In their wave runup test, \citet{ge2009} use a smarter approach using Clenshaw-Curtis quadrature integration, see their equation (4.17).

\bibliographystyle{plainnat}
\bibliography{swe-pc}


\end{document}

