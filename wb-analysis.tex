\documentclass{article}
\usepackage{fullpage}
\usepackage[utf8]{inputenc}
\usepackage[colorlinks,linkcolor=blue,citecolor=blue,urlcolor=blue]{hyperref}
\usepackage[round]{natbib}
\usepackage{doi}
\usepackage{amsmath}
\usepackage{amssymb}
\usepackage{mathtools}
\usepackage{xcolor}
\usepackage{bm}

\title{Well-balancing analysis}
\author{James Shaw}

\newcommand{\riemannflux}{\widetilde{\vect{F}}}
\newcommand{\source}{\vect{S}}
\newcommand{\vect}{\mathbf}
\newcommand{\zstar}{z^\star}

\begin{document}
For a lake-at-rest $u=0$ and $h + z = \text{constant}$.  We wish to show that the spatial operator is zero so that the lake remains at rest.
For the deterministic FV1 model with the surface gradient method, the spatial operator $\vect{L}$ is
\begin{align}
\vect{L} = \frac{1}{\Delta x} \left( \riemannflux_{i+1/2} - \riemannflux_{i-1/2} \right) - \source_i\\
\end{align}
The water elevation $\eta$ is constant and so $h_{i+1/2}^- = \eta - \zstar_{i+1/2}$, $h_{i+1/2}^+ = \eta - \zstar_{i+1/2}$, hence $h_{i+1/2}^- = h_{i+1/2}^+ = h_{i+1/2}$.
Since $q = hu = 0 $ then the mass conservation equation vanishes and the steady-state momentum equation is
\begin{align}
	\frac{\partial F_\mathrm{momentum}}{\partial x} - S_\mathrm{momentum} = 0 \\
	S_\mathrm{momentum} = - \frac{g}{\Delta x} \bar{h}_i \Delta z_i \\
	\frac{\partial F_\mathrm{momentum}}{\partial x} &\approx \frac{g}{2\Delta x} \left( h_{i+1/2}^2 - h_{i-1/2}^2 \right) \\
	&= \frac{g}{2 \Delta x} \left( h_{i+1/2} + h_{i-1/2} \right) \left( h_{i+1/2} - h_{i-1/2} \right) \\
	&= \frac{g}{\Delta x} \frac{h_{i+1/2} + h_{i-1/2}}{2} \left( \eta - \zstar_{i+1/2} - \eta + \zstar_{i-1/2} \right) \\
	&= - \frac{g}{\Delta x} \bar{h}_i \left( \zstar_{i+1/2} - \zstar_{i-1/2} \right) \\
	&= - \frac{g}{\Delta x} \bar{h}_i \Delta z_i
\end{align}

\end{document}
