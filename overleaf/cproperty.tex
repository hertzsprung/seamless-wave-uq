\subsection{Well-balanced Property with Uncertain Topography}

In this section it is shown that the stochastic Galerkin model satisfies the well-balanced property for a lake-at-rest with uncertain topography.
Assuming a lake-at-rest, the mean free-surface elevation is constant and the mean discharge is zero.
There is no uncertainty in the free-surface elevation or discharge.
The bed elevation can have any spatial profile and stochastic profile.

Since the free-surface elevation is constant and the averaged bed elevation is continuous at interfaces then $\eta_{i+1/2,p}^- - z^\star_{i+1/2,p} = \eta_{i+1/2,p}^+ - z^\star_{i+1/2,p}$, hence $h^{-,\star}_{i+1/2,p} = h^{+,\star}_{i+1/2,p}$ for all $i = 0, \ldots, M$ and $p = 0, \ldots, P$.
Shorthand notation is introduced such that
$\eta_p$ is the $p$\textsuperscript{th} expansion coefficient of the spatially uniform free-surface elevation, and
$h^\star_{i+1/2,p} = h^{-,\star}_{i+1/2,p} = h^{+,\star}_{i+1/2,p}$.
Since $h^\star_{i+1/2,p}$ is continuous at interfaces and the discharge is zero then the numerical flux is equal to the physical flux.

In order to preserve a lake-at-rest solution in element $i$ then the ensemble average of the flux gradient must balance the ensemble average of the source term vector in equation~\eqref{eqn:swe-pc}.
The balance of mass and momentum components can be considered separately.
The mass continuity equation balances because the discharge is zero.
For the momentum equation, the following equality must hold for all $l = 0, \ldots, P$:
\begin{align}
\frac{\Ensemble{F_{i+1/2}^{(n)} \pcbasis_l} - \Ensemble{F_{i-1/2}^{(n)} \pcbasis_l}}{\Delta x}
-
\Ensemble{S_i^{(n)} \pcbasis_l}
= 0
\label{eqn:momentum-balance-separate}
\end{align}
where the ensemble average of the bed slope source term $\Ensemble{S_i^{(n)} \pcbasis_l}$ is given by equation~\eqref{eqn:pc-source} and the momentum flux $F_{i+1/2}^{(n)}$ is
\begin{align}
F_{i+1/2}^{(n)} = \frac{g}{2}
\left\{\sum_{p=0}^P h^\star_{i+1/2,p} \pcbasis_p\right\}
\left\{\sum_{s=0}^P h^\star_{i+1/2,s} \pcbasis_s\right\}
\end{align}
and similarly for $F_{i-1/2}^{(n)}$.
Due to the distributive property of the basis (equation~\ref{eqn:distributive}), equation~\eqref{eqn:momentum-balance-separate} can be rewritten as
\begin{align}
\Ensemble{\frac{F_{i+1/2}^{(n)} 
-
F_{i-1/2}^{(n)}}{\Delta x} \pcbasis_l}
=
\Ensemble{ S_i^{(n)} \pcbasis_l}
\label{eqn:momentum-balance}
\end{align}
The ensemble average of the momentum flux gradient is
\begin{align}
    \Ensemble{\frac{F_{i+1/2}^{(n)} 
    -
    F_{i-1/2}^{(n)}}{\Delta x} \pcbasis_l}
    &=
    \frac{g}{2 \Delta x}
    \sum_{p=0}^P \sum_{\palt=0}^P
    \left( h^\star_{i+1/2,p} \, h^\star_{i+1/2,\palt} - 
    h^\star_{i-1/2,p} \, h^\star_{i-1/2,\palt} \right)
    \Ensemble{\pcbasis_p \pcbasis_\palt \pcbasis_l}
    \label{eqn:momentum-flux-unfactorised}
%
\intertext{To be able to factorise equation~\eqref{eqn:momentum-flux-unfactorised}, the first step is to add then subtract the term $h^\star_{i-1/2,p} h^\star_{i+1/2,s}$ to yield}
%
    \Ensemble{\frac{F_{i+1/2}^{(n)} 
    -
    F_{i-1/2}^{(n)}}{\Delta x} \pcbasis_l}
    &=
    \frac{g}{2 \Delta x}
    \sum_{p=0}^P \sum_{\palt=0}^P
    \left( h^\star_{i+1/2,p} \, h^\star_{i+1/2,\palt} - 
    h^\star_{i-1/2,p} \, h^\star_{i-1/2,\palt} + \right. \nonumber \\
    &\left.
    \hspace{6em}
    h^\star_{i-1/2,p} \, h^\star_{i+1/2,s} -
    h^\star_{i-1/2,p} \, h^\star_{i+1/2,s} \right)
    \Ensemble{\pcbasis_p \pcbasis_\palt \pcbasis_l}
    \label{eqn:momentum-flux-4term}
%
\intertext{Using the distributive property of multiplication, and associativity of summation, equation~\eqref{eqn:momentum-flux-4term} can be rewritten as}
%
    \Ensemble{\frac{F_{i+1/2}^{(n)} 
    -
    F_{i-1/2}^{(n)}}{\Delta x} \pcbasis_l}
    &=
    \frac{g}{2 \Delta x}
    \left[
    \sum_{p=0}^P \sum_{\palt=0}^P
    \left( h^\star_{i+1/2,p} \, h^\star_{i+1/2,\palt} - 
    h^\star_{i-1/2,p} \, h^\star_{i-1/2,\palt} + 
    h^\star_{i-1/2,p} \, h^\star_{i+1/2,s}
    \right)
    \Ensemble{\pcbasis_p \pcbasis_\palt \pcbasis_l}
    \right.
    \nonumber \\
    &\left.
    \hspace{2.5em}
    - 
    \sum_{p=0}^P \sum_{\palt=0}^P
    h^\star_{i-1/2,p} \, h^\star_{i+1/2,s}
    \Ensemble{\pcbasis_p \pcbasis_\palt \pcbasis_l} \right]
    \label{eqn:momentum-flux-split}
%
\intertext{
The final term in equation~\eqref{eqn:momentum-flux-split} can be rewritten as $\sum_{p=0}^P \sum_{s=0}^P h^\star_{i-1/2,s} \, h^\star_{i+1/2,p} \Ensemble{\pcbasis_p \pcbasis_s \pcbasis_l}$ due to commutativity of summation and product operators as well as the basis function (equation~\ref{eqn:commutative}).
Then, by associativity of summation:
}
%
    \Ensemble{\frac{F_{i+1/2}^{(n)} 
    -
    F_{i-1/2}^{(n)}}{\Delta x} \pcbasis_l}
    &=
    \frac{g}{2 \Delta x}
    \sum_{p=0}^P \sum_{\palt=0}^P
    \left( h^\star_{i+1/2,p} \, h^\star_{i+1/2,\palt} - 
    h^\star_{i-1/2,p} \, h^\star_{i-1/2,\palt} + \right. \nonumber \\
    &\left.
    \hspace{6em}
    h^\star_{i-1/2,p} \, h^\star_{i+1/2,s} -
    h^\star_{i-1/2,s} \, h^\star_{i+1/2,p} \right)
    \Ensemble{\pcbasis_p \pcbasis_\palt \pcbasis_l}
    \label{eqn:momentum-flux-4term-reordered}
%
\intertext{Finally, equation~\eqref{eqn:momentum-flux-4term-reordered} can be factorised as}
%
    \Ensemble{\frac{F_{i+1/2}^{(n)} 
    -
    F_{i-1/2}^{(n)}}{\Delta x} \pcbasis_l}
    &= 
    \frac{g}{2 \Delta x}
    \sum_{p=0}^P \sum_{\palt=0}^P
    \left( h^\star_{i-1/2,p} + h^\star_{i+1/2,p} \right)
    \left( h^\star_{i+1/2,s} - h^\star_{i-1/2,s} \right)
    \Ensemble{\pcbasis_p \pcbasis_\palt \pcbasis_l} \\
%
\intertext{and, given that $h^\star_{i+1/2,s} = \eta_s - \zmodified_{i+1/2,s}$ and $h^\star_{i-1/2,s} = \eta_s - \zmodified_{i-1/2,s}$, then}
    \Ensemble{\frac{F_{i+1/2}^{(n)} 
    -
    F_{i-1/2}^{(n)}}{\Delta x} \pcbasis_l}
    &=
    -\frac{g}{\Delta x}
    \sum_{p=0}^P \sum_{\palt=0}^P
    \left(
    \frac{h^{+,\star}_{i-1/2,p} + h^{-,\star}_{i+1/2,p}}{2}
    \right)
    \left( \zmodified_{i+1/2,\palt} - \zmodified_{i-1/2,\palt} \right)
    \Ensemble{\pcbasis_p \pcbasis_\palt \pcbasis_l}
    \label{eqn:momentum-flux}
\end{align}
Equation~\eqref{eqn:momentum-flux} is equal to the ensemble average of the bed slope source term $\Ensemble{S_i^{(n)} \pcbasis_l}$ given in equation~\eqref{eqn:pc-source}, hence discrete balance is preserved.
