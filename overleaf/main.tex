%\documentclass[Journal,letterpaper,InsideFigs,SingleSpace,NoLineNumbers]{ascelike-new}
\documentclass[Journal,letterpaper]{ascelike-new}
\usepackage[colorlinks,linkcolor=black,citecolor=black,urlcolor=black,draft=false]{hyperref}
\usepackage[utf8]{inputenc}
\usepackage[T1]{fontenc}
\usepackage{lmodern}
\usepackage{graphicx}
\usepackage[figurename=Fig.,labelfont=bf,labelsep=period]{caption}
\usepackage{subcaption}
\usepackage{amsmath}
\usepackage{amssymb}
\usepackage{bm}
\usepackage{doi}
\usepackage{mathtools}
\usepackage{newtxtext,newtxmath}
\usepackage[round]{natbib}
\usepackage{siunitx}
\usepackage{xcolor}
\usepackage{fancyvrb}
\usepackage{cprotect}
\usepackage{booktabs}

\NameTag{Shaw, \today}

\newcommand*\patchAmsMathEnvironmentForLineno[1]{%
  \expandafter\let\csname old#1\expandafter\endcsname\csname #1\endcsname
  \expandafter\let\csname oldend#1\expandafter\endcsname\csname end#1\endcsname
  \renewenvironment{#1}%
     {\linenomath\csname old#1\endcsname}%
     {\csname oldend#1\endcsname\endlinenomath}}% 
\newcommand*\patchBothAmsMathEnvironmentsForLineno[1]{%
  \patchAmsMathEnvironmentForLineno{#1}%
  \patchAmsMathEnvironmentForLineno{#1*}}%
\AtBeginDocument{%
\patchBothAmsMathEnvironmentsForLineno{equation}%
\patchBothAmsMathEnvironmentsForLineno{align}%
\patchBothAmsMathEnvironmentsForLineno{flalign}%
\patchBothAmsMathEnvironmentsForLineno{alignat}%
\patchBothAmsMathEnvironmentsForLineno{gather}%
\patchBothAmsMathEnvironmentsForLineno{multline}%
}

\DeclareMathOperator{\E}{\mathbb{E}}
\DeclareMathOperator{\sech}{sech}
\newcommand{\anyvar}{\alpha}
\newcommand{\dee}{\mathrm{d}}
\newcommand{\Deltazhat}{\widehat{\Delta z}}
\newcommand{\diff}{\:\dee}
\newcommand{\Ensemble}[1]{\left\langle #1 \right\rangle}
\newcommand{\etamean}{\mean{\eta}}
\newcommand{\flow}{\vect{U}}
\newcommand{\floworbed}{u}
\newcommand{\flowKmodified}{\flow^{K,\star}}
\newcommand{\flux}{\vect{F}}
\newcommand{\hhat}{\hat{h}}
\newcommand{\hmodified}{h^\star}
\newcommand{\hKmodified}{h^{K,\star}}
\newcommand{\hump}{r}
\newcommand{\humpmean}{\mean{r}}
\newcommand{\Mag}[1]{\left\lvert #1 \right\rvert}
\newcommand{\mean}[1]{\overline{#1}}
\newcommand{\palt}{s}
\newcommand{\pcbasis}{\Phi}
\newcommand{\pcbasisvect}{\bm{\Phi}}
\newcommand{\qKmodified}{q^{K,\star}}
\newcommand{\randomroot}{\hat{\xi}}
\newcommand{\riemannflux}{\vect{\tilde{F}}}
\newcommand{\riemannfluxcomponent}{\tilde{F}}
\newcommand{\source}{\vect{S}}
\newcommand{\T}{\intercal}
\newcommand{\vect}{\mathbf}
\newcommand{\velocity}{v}
\newcommand{\zmean}{\mean{z}}
\newcommand{\zmodified}{z^\star}
\newcommand{\zrectbump}{\breve{z}}

\newcommand{\rev}[1]{\textcolor{orange}{#1}}

\author[1]{James Shaw, Ph.D}
\author[2]{Georges Kesserwani, Ph.D}

\affil[1]{Research Associate, Department of Civil and Structural Engineering, The University of Sheffield, Western Bank, Sheffield S10 2TN, U.K. Email: js102@zepler.net}
\affil[2]{Research Fellow, Department of Civil and Structural Engineering, The University of Sheffield, Western Bank, Sheffield S10 2TN, U.K. Email: g.kesserwani@sheffield.ac.uk}

\title{Stochastic Galerkin finite volume shallow flow model: well-balanced treatment over uncertain topography}

\begin{document}

\maketitle

\begin{abstract}
Stochastic Galerkin methods can quantify uncertainty at a fraction of the computational expense of conventional Monte Carlo techniques, but such methods have rarely been studied for modelling shallow water flows.
Existing stochastic shallow flow models are not well-balanced and their assessment has been limited to stochastic flows with smooth probability distributions.
This paper addresses these limitations by formulating a one-dimensional stochastic Galerkin shallow flow model using a low-order Wiener-Hermite Polynomial Chaos expansion with a finite volume Godunov-type approach, incorporating the surface gradient method to guarantee well-balancing.
Preservation of a lake-at-rest over uncertain topography is verified analytically and numerically.
The model is also assessed using flows with discontinuous and highly non-Gaussian probability distributions.
Prescribing constant inflow over uncertain topography, the model converges on a steady-state flow that is subcritical or transcritical depending on the topography elevation.
Using only four Wiener-Hermite basis functions, the model produces probability distributions comparable to those from a Monte Carlo reference simulation with 2000 iterations, while executing about 100 times faster.
Accompanying model software and simulation data is openly available online.
\end{abstract}

\clearpage

%This paper addresses these limitations by formulating a well-balanced one-dimensional stochastic Galerkin shallow flow model using a low-order Wiener-Hermite \rev{Polynomial Chaos} basis.
%The model is based on a stochastic reformulation of a finite volume Godunov-type approach, incorporating the surface gradient method to guarantee well-balancing.
%Despite its relatively low-order \rev{Wiener-Hermite} basis, the model produces probability distributions comparable to those from a Monte Carlo reference simulation, and executes about 100 times faster.
\section{Introduction}
Shallow water flows can be highly sensitive to uncertainties in input data: uncertainties in boundary forcing derived from stream gauge data, the choice of friction coefficient, and errors in digital elevation models can substantially alter simulated distributions of flood extent, water depth and flow velocity \citep{bates2014,jung-merwade2012}.
Conventional uncertainty quantification methods, including the Generalized Likelihood Uncertainty Estimation (GLUE) method \citep{beven-binley1992}, are based on Monte Carlo sampling with randomised inputs, making conventional methods computationally expensive and slow to converge on flow statistics.
As a result, Monte Carlo simulations of flood events are severely constrained by available computing resources, and many sources of uncertainty must be neglected to make probabilistic simulations feasible \citep{neal2013}.
Stochastic Galerkin methods offer a computationally efficient alternative: repeated sampling is eliminated, and convergence is typically far more rapid than Monte Carlo methods \citep{xiu2009,ge2008}.
%Despite widespread interest in stochastic Galerkin and other polynomial chaos methods, existing stochastic Galerkin shallow water models have lacked any well-balanced treatment, and have only been validated for flows with smooth probability distributions \citep{ge2008}.
%This paper addresses these limitations by formulating a well-balanced stochastic Galerkin shallow water model, and assessing the model using flows with discontinuous, non-Gaussian probability distributions.

Polynomial Chaos methods work by introducing to the governing deterministic equations an additional, stochastic dimension, and approximating it with a Polynomial Chaos expansion.
Polynomial Chaos methods can be classified as non-intrusive or intrusive.
Non-intrusive methods repeatedly sample a deterministic model with different input values, using the outputs to construct a stochastic solution.
While non-intrusive polynomial chaos methods involve repeated sampling, they require far fewer samples than Monte Carlo methods \citep{ge2008}.
Intrusive methods reformulate a deterministic model to produce a stochastic formulation that must be implemented as a new stochastic solver.
The stochastic Galerkin method is an intrusive method which makes a Galerkin projection in stochastic space to produce a decoupled system of equations that are solved in a single model run.
Conventional stochastic Galerkin methods span the entire stochastic space using a single element with a basis chosen from the Askey scheme of orthogonal polynomials \citep{xiu-karniadakis2002}.
When the stochastic flow is sufficiently smooth and the probability distributions are well-represented by the Polynomial Chaos basis, then Polynomial Chaos methods can achieve exponential convergence in the stochastic dimension \citep{xiu-karniadakis2003}.

Conventional stochastic Galerkin methods have two main shortcomings.
First, strongly nonlinear flows produce steep gradients or discontinuities in stochastic space that are often poorly represented by global Polynomial Chaos bases using a single element \citep{pettersson2014}.
Second, even when exponential convergence is achieved early in the simulation, solution accuracy can deteriorate severely over long simulations \citep{gerritsma2010}.
These shortcomings have motivated a wealth of research into better alternatives, including time-dependent basis functions \citep{gerritsma2010}, multiresolution wavelet decompositions \citep{lemaitre2004a}, multi-element discretisations and adaptive meshing of stochastic space \citep{wan-karniadakis2006,tryoen2010a,pettersson2014,li-stinis2015}.
%Similar adaptive methods have also been developed for non-intrusive stochastic collocation methods, and these are reviewed by \citet{bhaduri2018}.

While stochastic Galerkin methods have received a great deal of attention, few have applied the method to the shallow water equations.
\citet{ge2008} made the first step in this direction, formulating a one-dimensional stochastic Galerkin shallow water model to study solitary wave propagation over uncertain topography.
Their stochastic Galerkin model was 50 times more efficient than Monte Carlo with similar accuracy, demonstrating that stochastic Galerkin shallow water models offer a viable alternative to Monte Carlo methods.
However, their experiments were limited to flows with smooth, unimodal probability distributions that are more easily represented by stochastic Galerkin methods.
Furthermore, their stochastic Galerkin formulation used a centred difference approximation of the bed slope source term that is not well-balanced.

A well-balanced stochastic Galerkin model was formulated by \citet{jin2016} and applied to scalar nonlinear equations including the inviscid Burgers' equation.
The model inherited the well-balancing property from the underlying deterministic model that used the `interface method' by \citet{jin2001}.
\citet{jin2016} note that the intrusive stochastic Galerkin method has a particular advantage over non-intrusive stochastic collocation:
a stochastic collocation method can only guarantee well-balancing at the collocation nodes where the deterministic well-balanced model is sampled.
In contrast, the stochastic Galerkin model guarantees well-balancing across the entire stochastic space, irrespective of the order of Polynomial Chaos basis.

When applied to the shallow water equations, the well-balanced interface method is equivalent to the surface gradient method by \citet{zhou2001}.
A stochastic reformulation of the surface gradient method is straightforward because it is a linear method involving only linear combinations of the discrete flow variables and topography.
Popular alternatives to the surface gradient method include well-balanced approaches that are also depth-positivity preserving \citep{audusse2004,liang-marche2009}.
Such approaches use nonlinear $\max(\cdot, \cdot)$ operators which greatly complicate their stochastic Galerkin reformulation for realistic applications.

This paper presents a well-balanced stochastic Galerkin model of the one-dimensional shallow water equations using a stochastic reformulation of the surface gradient method \citep{zhou2001}.
The well-balancedness of the proposed stochastic Galerkin model is studied theoretically and verified numerically for a lake-at-rest flow over an uncertain bed.
Next, a steady-state test is proposed that provokes a strongly nonlinear flow response over an uncertain bed, resulting in probability distributions that are highly non-Gaussian and discontinuous.
Probability distributions produced by the stochastic Galerkin model are validated against a Monte Carlo reference simulation, and the relative computational performance is discussed.
Numerical simulation data \citep{shaw-kesserwani2018a} and a Python 3 implementation of the stochastic Galerkin shallow flow model \citep{shaw-kesserwani2018b} are available to download from Zenodo.
Instructions for running the models and interpreting the data are provided in the appendix.

%This paper makes the next step by formulating a well-balanced stochastic Galerkin shallow water model, and assessing the model using flows with discontinuous, non-Gaussian probability distributions.

\clearpage
\section{Deterministic and stochastic shallow flow models}

In this section, a certain deterministic numerical solver of the one dimensional (1D) shallow water equations is outlined in the framework of a finite volume Godunov-type method.
The selected deterministic solver relies on the surface gradient method \citep{zhou2001} to both ensure a well-balanced topography integration and extendibility of the well-balanced property into the stochastic Galerkin case. 
Accordingly, a stochastic Galerkin reformulation is devised that is theoretically well-balanced with uncertain topography under a lake-at-rest hypothesis.

The mathematical model of the shallow water equations represent mass and momentum
conservation principals, and is used in the following conservative form when solving it within a finite volume Godunov-type framework \citep{toro-garcianavarro2007}:
\begin{align}
\frac{\partial \flow(x, t)}{\partial t} + \frac{\partial \flux(\flow(x, t))}{\partial x} = \source(\flow(x, t), z(x)) \label{eqn:swe}
\end{align}
where $\flow = \left[ h, q \right]^\T$ is the flow vector including water height $h$ ($\mathrm{L}$) and unit-width discharge $q = h\velocity$ ($\mathrm{L}^2/\mathrm{T}$) in which $\velocity$ represents the depth-averaged velocity ($\mathrm{L}/\mathrm{T}$), $\flux = \left[ q,  q^2/h + gh^2/2 \right]^\T$ is the flux vector in which $g$ represents the gravitational constant and $\source = \left[ 0, -gh \: \dee z / \dee x \right]^\T$ is the source term vector in which the gradient of the topography $z(x)$ is involved.

\subsection{Deterministic model}

On a uniform 1D mesh with $M$ elements each of size $\Delta x$, the first-order finite volume method leads to the following discrete element-wise formulation of equation~\eqref{eqn:swe}:
\begin{align}
    \flow_i^{(n+1)} = \flow_i^{(n)} - \Delta t
    \left(
    \frac{\riemannflux_{i+1/2}^{(n)} - \riemannflux_{i-1/2}^{(n)}}{\Delta x}
    - \source_i^{(n)} \right) \label{eqn:swe-discrete}
\end{align}
in which $\flow_i^{(n)} = \left[ h_i^{(n)}, q_i^{(n)} \right]^\T$ is a piecewise-constant discretisation of the flow vector at element $i$ and time level $(n)$, and $\riemannflux_{i+1/2}^{(n)}$ is a numerical flux function for linking nonlinear discontinuities associated with the flow vector data at interface $i+1/2$ located between element $i$ and $i + 1$.
Namely, $\riemannflux_{i+1/2}^{(n)} = \riemannflux(\flow_{i+1/2}^-, \flow_{i+1/2}^+)$ where $\flow_{i+1/2}^-$ is the limit of the solution from the side of element $i$, and $\flow_{i+1/2}^+$ is the limit of the solution from the side of element $i+1$.
Within the scope of this work involving a first-order accurate solver, these limits become $\flow_{i+1/2}^- = \flow_i$ and $\flow_{i+1/2}^+ = \flow_{i+1}$ that are used in a numerical flux function based on the approximate Riemann solver of Roe \citep{roe-pike1984}.

The surface gradient method essentially reconstructs an averaged topography at interface $i+1/2$ that is shared by both elements $i$ and $i+1$, as $\zmodified_{i+1/2} = (z_i + z_{i+1})/2$.
From the reconstructed topography $\zmodified_{i+1/2}$, consistent flow variable limits are accordingly reconstructed based on the actual free-surface elevation data, i.e. $\eta_{i+1/2}^- = h_i^{(n)} + z_i$ and $\eta_{i+1/2}^+ = h_{i+1}^{(n)} + z_{i+1}$, and velocity data, i.e. $\velocity_{i+1/2}^- = q_i^{(n)}/h_i^{(n)}$ and $\velocity_{i+1/2}^+ = q_{i+1}^{(n)} / h_{i+1}^{(n)}$, as: $\hKmodified_{i+1/2} = \eta_{i+1/2}^K - \zmodified_{i+1/2}$ and $\qKmodified_{i+1/2} = \hKmodified_{i+1/2} \velocity_{i+1/2}^K$ (where $K = + \text{ or } -$).
For clarity of presentation, the time level denoted by superscript $(n)$ is omitted from all reconstructed variables.
These reconstructions form new Riemann states $\flowKmodified_{i+1/2} = \left[ \hKmodified_{i+1/2}, \qKmodified_{i+1/2} \right]^\T$ for use to evaluate $\riemannflux_{i+1/2}^{(n)}$.
By analogy, new Riemann limits $\flowKmodified_{i-1/2} = \left[ \hKmodified_{i-1/2}, \qKmodified_{i-1/2} \right]^\T$ at $i - 1/2$ are produced for use to evaluate $\riemannflux_{i-1/2}^{(n)}$.
From the reconstructed limits, a well-balanced discretisation of the source term vector can be produced:
\begin{align}
	\source_i^{(n)} = \left[ 0, -g
	\left( \frac{h^{+,\star}_{i-1/2} + h^{-,\star}_{i+1/2}}{2} \right)
	\left( \frac{z^\star_{i+1/2} - z^\star_{i-1/2}}{\Delta x} \right)
	\right]^\T
	\label{eqn:source}
\end{align}
The well-balanced deterministic model presented in equations~\eqref{eqn:swe-discrete} and \eqref{eqn:source} is used for Monte Carlo simulations, and the deterministic model is also the starting point for a stochastic Galerkin reformulation.

\subsection{Fundamental properties of the Polynomial Chaos basis}

Before presenting the stochastic Galerkin reformulation, it is necessary to consider a single random variable $X(\theta)$ that maps from the random event $\theta$ to an arbitrary probability distribution with finite variance.
This random variable can be approximated by a Wiener-Hermite Polynomial Chaos expansion \citep{xiu-karniadakis2002}.
The expansion is based on a standard Gaussian random variable $\xi(\theta) \in [-\infty, +\infty]$ having zero mean and unit variance.
The random variable of interest, $X(\theta)$, is then approximated as
\begin{align}
X(\theta) \approx \sum_{p=0}^P X_p \pcbasis_p(\xi(\theta))
\end{align}
where $\vect{X} = \left[ X_0, \ldots, X_P \right]^\T$ are the expansion coefficients and $\pcbasisvect = \left[ \pcbasis_0, \ldots, \pcbasis_P \right]^\T$ is the probabilists' Hermite polynomial basis having basis function $\pcbasis_p$ of degree $p$,
\begin{align}
    \pcbasis_p(\xi) = \left( -1 \right)^p \exp \left(\frac{\xi^2}{2}\right)
    \frac{\dee^p}{\dee \xi^p} \exp \left(- \frac{\xi^2}{2} \right)
\end{align}
where $\pcbasis_0 = 1, \pcbasis_1 = \xi, \pcbasis_2 = \xi^2 - 1, \pcbasis_3 = \xi^3 - 3\xi$ and so on.
As the basis order $P$ is increased, the Wiener-Hermite Polynomial Chaos approximation converges on the true random variable $X(\theta)$ \citep{xiu-karniadakis2002}.

\subsubsection*{Basis orthogonality and commutativity}
The Wiener-Hermite basis $\pcbasisvect$ is orthogonal such that
\begin{align}
	\Ensemble{\pcbasis_p \pcbasis_s} = \Ensemble{\pcbasis_p^2} \delta_{ps}
\end{align}
where $\Ensemble{\cdot}$ is the ensemble average operator and $\delta_{ps}$ is the Kronecker delta that is equal to one when $p = s$ and zero otherwise.
The ensemble average operator is defined as the weighted integral over the standard Gaussian random variable $\xi$:
\begin{align}
	\Ensemble{\alpha(\xi)} = \int_{-\infty}^\infty \alpha(\xi) W(\xi) \diff \xi \label{eqn:ensemble-average}
\end{align}
where $\alpha(\xi)$ is an expression involving any combination of random variables or basis functions, and the weighting function $W(\xi)$ is the standard Gaussian probability density function
\begin{align}
	W(\xi) = \frac{1}{\sqrt{2\pi}} \exp \left(-\frac{\xi^2}{2}\right)
\end{align}
This weighting function ensures that, when $\alpha$ is independent of $\xi$, the ensemble average $\Ensemble{\alpha} = \alpha$.
Finally, the ensemble average of a product of basis functions is commutative such that
\begin{align}
    \Ensemble{\pcbasis_p \pcbasis_s} = \Ensemble{\pcbasis_s \pcbasis_p}
    \label{eqn:commutative}
\end{align}
The commutative property is needed later when verifying the well-balanced property with uncertain topography.

\subsubsection*{Stochastic Galerkin projection of a random variable}
Given the random variable $X(\theta) = \sum_{p=0}^P X_p \pcbasis_p(\xi(\theta))$, its Galerkin projection onto a basis function $\pcbasis_l$ with $l = 0, \ldots, P$ is achieved using the ensemble average operator such that, due to orthogonality,
\begin{align}
	\Ensemble{X(\theta) \pcbasis_l} = X_l \Ensemble{\pcbasis_l^2} \label{eqn:orthogonal}
\end{align}
where $X_l$ is the $l$\textsuperscript{th} order expansion coefficient.
Also note that the Galerkin projection of a basis function $\pcbasis_l$ and two random variables, $X(\theta)$ and $Y(\theta)$, is distributive:
\begin{align}
	\Ensemble{\left(X(\theta) + Y(\theta)\right) \pcbasis_l}
	=
	\Ensemble{X(\theta) \pcbasis_l} + \Ensemble{Y(\theta) \pcbasis_l} \label{eqn:distributive}
\end{align}

\subsubsection*{Mean, variance and high-order moments}
The mean, variance and high-order moments can be calculated for $X(\theta)$.
The $m$\textsuperscript{th} moment $\mu_m[X]$ is defined as
\begin{align}
\mu_m[X] = \int_{-\infty}^\infty \left(X - \beta \right)^m W(\xi) \diff \xi
    =
    \Ensemble{\left( X - \beta \right)^m} \label{eqn:moment}
\end{align}
where $\beta = 0$ when $m = 1$ and $\beta = \mu_1[X]$ for higher-order moments.
Therefore, the mean $\mu_1[X] = \Ensemble{X} = \sum_{p=0}^P X_p \Ensemble{\pcbasis_p}$.
Since $\Ensemble{\pcbasis_0} = 1$ and $\Ensemble{\pcbasis_p} = 0$ for $p > 0$ then
\begin{align}
\mu_1[X] = X_0
\label{eqn:mean}
\end{align}
The shorthand notation for the mean of $X$ is $\mean{X}$, also known as the expected value, $\E\left[X\right]$.

The variance $\mu_2[X]$ can be derived using the fact that $\E\left[ \left( X - \E[X] \right)^2 \right] = \E[X^2] - \E^2[X]$, hence $\mu_2[X] = \left(\sum_{p=0}^P X_p^2 \Ensemble{\pcbasis_p^2}\right) - X_0^2 \Ensemble{\pcbasis_0}^2$.
Since $\Ensemble{\pcbasis_0}^2 = \Ensemble{\pcbasis_0^2}$ then
\begin{align}
    \mu_2[X] &= \sum_{p=1}^P X^2_p \Ensemble{\pcbasis_p^2} \label{eqn:variance}
\end{align}
The shorthand notation for the variance of $X$ is $\sigma^2_X$ and the standard deviation of $X$ is $\sigma_X$.

\subsubsection*{Reconstructing the probability density function}
The probability density function $f_X(x)$ of a random variable $X$ is,
\begin{subequations}
\begin{align}
        f_X(x) = \sum_{j=1}^J \Mag{ \sum_{p=0}^P X_p \frac{\dee \Phi_p}{\dee \xi}(\randomroot_j)}^{-1} W(\randomroot_j)
%
\intertext{where $\randomroot_j$, $j=1, \ldots, J$ are the real roots of the polynomial}
%
        x - \sum_{p=0}^P X_p \pcbasis_p(\xi) = 0
\end{align}\label{eqn:pdf}%
\end{subequations}
which can be calculated numerically for a specific outcome $x$.
Hence, the probability density function is computed by evaluating equation~\eqref{eqn:pdf} for a range of outcomes.

\subsection{Stochastic Galerkin reformulation of the deterministic model}
%In the deterministic 1D shallow water equations (equation~\ref{eqn:swe}), the flow vector is $\flow(x, t)$ and the topography is $z(x)$.
The solution of the stochastic 1D shallow water equations is now random because it depends on uncertain initial conditions, uncertain boundary conditions and uncertain topography.
Hence, the stochastic 1D shallow water equations depend not only upon space $x$ and time $t$, but additionally upon the random event $\theta$.
The stochastic flow vector $\flow(x, t, \theta)$ becomes a general stochastic process having arbitrary probability distributions that vary in space and time.
Similarly, the stochastic topography $z(x, \theta)$ has arbitrary probability distributions that vary in space.

The stochastic Galerkin reformulation of the deterministic model involves three steps to (i) replace the deterministic variables, $\flow_i^{(n)}$ and $z_i$, with random variables $\flow_i^{(n)}(\theta)$ and $z_i(\theta)$, (ii) rewrite the deterministic formulation using these random variables, and (iii) make a stochastic Galerkin projection onto the Wiener-Hermite basis.

\subsubsection*{Replacing deterministic variables with random variables}

For all elements $i=1, \ldots, M$ across all time levels, every deterministic flow variable $\flow_i^{(n)} = \left[h_i^{(n)}, q_i^{(n)}\right]^\T$ and deterministic topography variable $z_i$ becomes a random variable approximated by a Wiener-Hermite Polynomial Chaos expansion:
\begin{align}
\flow_i^{(n)}(\theta) \approx \sum _{p=0}^P \flow_{i,p}^{(n)} \pcbasis_p(\xi(\theta))
    \:\text{,}\quad
z_i(\theta) \approx \sum_{p=0}^P z_{i,p} \pcbasis_p(\xi(\theta))
\label{eqn:pc-expansion}%
\end{align}
where $\flow_{i,p}^{(n)} = \left[ h_{i,p}^{(n)}, q_{i,p}^{(n)} \right]^\T$ and $z_{i,p}$ are the $p$\textsuperscript{th} order expansion coefficients over element $i$ at time level $n$.

The reconstructed topography and reconstructed limits become functions of random variables.
The reconstructed topography at interface $i+1/2$ becomes
\begin{align}
	\sum_{p=0}^P z^\star_{i+1/2,p} \pcbasis_p
	=
	\frac{1}{2}
	\left(
	\sum_{p=0}^P z_{i,p} \pcbasis_p
	+
	\sum_{p=0}^P z_{i+1,p} \pcbasis_p
	\right)
\end{align}
and so $z^\star_{i+1/2,p} = (z_{i,p} + z_{i+1,p})/2$ due to basis orthogonality.
The reconstructed limits $\flow^{K,\star}_{i+1/2,p} = \left[ h^{K,\star}_{i+1/2,p}, q^{K,\star}_{i+1/2,p} \right]^\T$ (where $K = + \text{ or } -$) are calculated in a similar fashion.

Random variables that are functions of other random variables can be calculated in the same way.
In particular, water height can be expressed as a function of free-surface elevation and topography such that, due to basis orthogonality,
\begin{align}
h_{i,p}^{(n)} = \eta_{i,p}^{(n)} - z_{i,p}
\label{eqn:h-eta-z}
\end{align}
which is used later for specifying initial conditions.

\subsubsection*{Rewriting the deterministic formulation using random variables}

The deterministic finite volume formulation given by equation~\eqref{eqn:swe-discrete} is rewritten in terms of the random variables in equation~\eqref{eqn:pc-expansion}.
As a result, the numerical fluxes $\riemannflux_{i+1/2}^{(n)}$ and $\riemannflux_{i-1/2}^{(n)}$ and source term vector $\source_i^{(n)}$ become functions of random variables.
The numerical flux $\riemannflux_{i+1/2}^{(n)}$ becomes
\begin{align}
	\riemannflux_{i+1/2}^{(n)} = \riemannflux \left(
	\sum_{p=0}^P \flow^{-,\star}_{i+1/2,p} \pcbasis_p, 
	\sum_{p=0}^P \flow^{+,\star}_{i+1/2,p} \pcbasis_p
	\right)
\end{align}
and similarly for $\riemannflux_{i-1/2}^{(n)}$.
The source term vector $\source_i^{(n)}$ in equation~\eqref{eqn:source} becomes
\begin{align}
	\source_i^{(n)} = \left[ 0, -\frac{g}{\Delta x}
	\left\{
	\sum_{p=0}^P \frac{h^{+,\star}_{i-1/2,p} + h^{-,\star}_{i+1/2,p}}{2} \pcbasis_p \right\}
\left\{ \sum_{s=0}^P \left( z^\star_{i+1/2,s} - z^\star_{i-1/2,s} \right) \pcbasis_s \right\}
	\right]^\T
	\label{eqn:random-source}
\end{align}
Equation~\eqref{eqn:random-source} involves the product of two expressions, each delimited by braces.
Since both expressions include Wiener-Hermite expansions then different indices, $p$ and $s$, are needed for the expansion coefficients in each expression.

\subsubsection*{Stochastic Galerkin projection}

Due to the orthogonal property (equation~\ref{eqn:orthogonal}) and distributive property (equation~\ref{eqn:distributive}) of the Wiener-Hermite basis, a Galerkin projection of equation~\eqref{eqn:swe-discrete} onto the basis functions $\pcbasis_l, l = 0, \ldots, P$ produces $P+1$ decoupled equations:
\begin{align}
    \flow_{i,l}^{(n+1)} = \flow_{i,l}^{(n)}
    - \frac{\Delta t}{\Ensemble{\pcbasis_l^2}}
    \left(
    \frac{
    \Ensemble{\riemannflux_{i+1/2}^{(n)} \pcbasis_l}
    -
    \Ensemble{\riemannflux_{i-1/2}^{(n)} \pcbasis_l}
    }{\Delta x}
    - \Ensemble{\source_i^{(n)} \pcbasis_l}
    \right) \label{eqn:swe-pc}
\end{align}
Equation~\eqref{eqn:swe-pc} involves ensemble averages of numerical fluxes, $\Ensemble{\riemannflux_{i+1/2}^{(n)} \pcbasis_l}$ and $\Ensemble{\riemannflux_{i-1/2}^{(n)} \pcbasis_l}$, and an ensemble average of the source term vector, $\Ensemble{\source_i^{(n)} \pcbasis_l}$.
There is no straightforward method for calculating an ensemble average of the numerical flux because it is nonlinear.
Instead, the integral in equation~\eqref{eqn:ensemble-average} is approximated by Gauss-Hermite quadrature,
\begin{align}
    \Ensemble{\riemannflux_{i+1/2}^{(n)} \pcbasis_l}
    \approx
    \sum_{j=1}^{P+1} w_j
    \riemannflux\left(
	\sum_{p=0}^P \flow_{i+1/2,p}^{-,\star} \pcbasis_p(\xi_j),
	\sum_{p=0}^P \flow_{i+1/2,p}^{+,\star} \pcbasis_p(\xi_j)
	\right)
    \pcbasis_l(\xi_j) W(\xi_j) \label{eqn:pc-flux}
\end{align}
where $w_j$ are the quadrature weights and $\xi_j$ are the quadrature points.
The ensemble average $\Ensemble{\riemannflux_{i-1/2}^{(n)} \pcbasis_l}$ is calculated in the same way.

Unlike the nonlinear numerical flux, the ensemble average of the source term vector $\Ensemble{\source_i^{(n)} \pcbasis_l}$ is linear and can be derived directly from equation~\eqref{eqn:random-source}:
\begin{align}
\Ensemble{\source_i^{(n)} \pcbasis_l} &= \left[ 0,
    - \frac{g}{\Delta x}
    \sum_{p=0}^P \sum_{s=0}^P
\left(\frac{h^{+,\star}_{i-1/2,p} + h^{-,\star}_{i+1/2,p}}{2}\right)
\left( z^\star_{i+1/2,s} - z^\star_{i-1/2,s} \right)
    \Ensemble{\pcbasis_p \pcbasis_s \pcbasis_l}
    \right]^\T
\label{eqn:pc-source}
\end{align}
The ensemble averages $\Ensemble{\pcbasis_p \pcbasis_s \pcbasis_l}$ in equation~\eqref{eqn:pc-source} and $\Ensemble{\pcbasis_l^2}$ in equation~\eqref{eqn:swe-pc} can be calculated analytically or exactly by Gauss-Hermite quadrature.
Since these calculations do not depend on the solution then they can be precomputed once and stored.

\section{Analytic verification of the stochastic Galerkin conservative property}

Using a deterministic shallow water model formulation, the surface gradient method satisfies the exact \cproperty{} \citep{zhou2001} such that the deterministic model exactly preserves an initial, stationary solution over a sloping bed.
In this section it is shown that the stochastic Galerkin model also satisfies the exact \cproperty.

The initial mean water elevation $\eta_0 = h_0 + z_0 $ is constant and the initial mean discharge $q_0 = 0$.
There is zero uncertainty in the initial water elevation and initial discharge.
The bed elevation $z$ is arbitrary, having any spatial and stochastic profile.
Since the water elevation is spatially uniform and the averaged bed elevation $\zmodified$ is continuous at interfaces then $h_{i+1/2}^- = h_{i+1/2}^+$ and the shorthand notation $h_{i+1/2}$ can be used.
The mass conservation equation vanishes since $q = 0$.
In order to preserve a stationary solution in element $i$ then $\partial q_i/\partial t = 0$ and the momentum equation becomes
\begin{align}
    \frac{\riemannfluxcomponent_{i+1/2} - \riemannfluxcomponent_{i-1/2}}{\Delta x} = S_i
    \label{eqn:momentum-balance}
\end{align}
where the numerical momentum flux $\riemannfluxcomponent_{i+1/2}$ and bed slope source term $S_i$ are
\begin{align}
\riemannfluxcomponent_{i+1/2} = \frac{g}{2} h_{i+1/2}^2 \text{\quad,\quad}
S_i = - \frac{g}{\Delta x} \hmodified_i \Delta \zmodified_i
\end{align}
To show that the stochastic Galerkin model formulation satisfies the exact \cproperty{}, it is sufficient to show that the stochastic Galerkin projection of equation~\eqref{eqn:momentum-balance} is satisfied.
The stochastic Galerkin projection of the source term $\Ensemble{S_i \pcbasis_l}$ is given by equation~\eqref{eqn:pc-source-h}.
The stochastic Galerkin projection of the momentum flux is
\begin{align}
    \Ensemble{\frac{\riemannfluxcomponent_{i+1/2} -
    \riemannfluxcomponent_{i-1/2}}{\Delta x}
    \pcbasis_l}
    &=
    \frac{g}{2 \Delta x}
    \sum_{p=0}^P \sum_{\palt=0}^P
    h_{i+1/2,p} \, h_{i+1/2,\palt} - 
    h_{i-1/2,p} \, h_{i-1/2,\palt} 
    \Ensemble{\pcbasis_p \pcbasis_\palt \pcbasis_l} \nonumber \\
%
\intertext{Since $h_{i+1/2,p}$ and $h_{i+1/2,\palt}$ are interchangeable and similarly for $h_{i-1/2,p}$ and $h_{i-1/2,\palt}$ then}
%
    &= 
    \frac{g}{2 \Delta x}
    \sum_{p=0}^P \sum_{\palt=0}^P
    \left( h_{i+1/2,p} + h_{i-1/2,p} \right)
    \left( h_{i+1/2,\palt} - h_{i-1/2,\palt} \right)
    \Ensemble{\pcbasis_p \pcbasis_\palt \pcbasis_l} \nonumber \\
%
\intertext{and, given that $h_{i+1/2,\palt} = \eta_\palt - \zmodified_{i+1/2,\palt}$ where the water elevation coefficient $\eta_\palt$ is spatially uniform, then}
    &=
    -\frac{g}{\Delta x}
    \sum_{p=0}^P \sum_{\palt=0}^P
    \frac{h_{i+1/2,p} + h_{i-1/2,p}}{2}
    \left( \zmodified_{i+1/2,\palt} - \zmodified_{i-1/2,\palt} \right)
    \Ensemble{\pcbasis_p \pcbasis_\palt \pcbasis_l} \nonumber \\
%
    \Ensemble{\frac{\riemannfluxcomponent_{i+1/2} -
    \riemannfluxcomponent_{i-1/2}}{\Delta x}
    \pcbasis_l}
    &=
    - \frac{g}{\Delta x}
    \sum_{p=0}^P \sum_{\palt=0}^P
    \hmodified_{i,p} \Delta \zmodified_{i,\palt}
    \Ensemble{\pcbasis_p \pcbasis_\palt \pcbasis_l}
    = \Ensemble{S_i \pcbasis_l}
\end{align}
which satisfies equation~\eqref{eqn:momentum-balance} as required.
\section{Numerical experiments}
Two numerical experiments are performed to assess the stochastic Galerkin shallow water models.
The first test simulates a lake at rest over a bed with a region of uncertainty.
This test assesses the ability of the stochastic Galerkin models to preserve the C-property.
The second test simulates flow over a hump of uncertain amplitude.
This test is designed to challenge the stochastic Galerkin models by generating a nonlinear response such that the steady-state solution may be subcritical or transcritical depending on the hump amplitude.

For both tests, the one-dimensional domain is [\SI{-50}{\meter}, \SI{50}{\meter}], tessellated by $M = 100$ elements such that $\Delta x = \SI{0.5}{\meter}$.
Both tests include a topographic hump centred at $x = \SI{0}{\meter}$.
Following a similar approach to \citet{ge2008}, a bed elevation $z$ with a Gaussian probability distribution is defined as
\begin{align}
z(x, a) = a \sech^2 \left( \frac{\pi x}{\lambda} \right) \label{eqn:bed}
\end{align}
where the uncertain hump amplitude $a$ has a mean value $\amean = \SI{0.6}{\meter}$ and standard deviation $\sigma_a = \SI{0.3}{\meter}$, and half-width $\lambda = \SI{10}{\meter}$.
Before calculating the bed elevation coefficients $z_{i,0}, \ldots, z_{i,P}$, the uncertain bed elevation profile given by equation~\eqref{eqn:bed} must be expressed in terms of mean bed elevation $\zmean(x)$ and bed elevation variance $\sigma_z^2(x)$.
The mean bed elevation is simply $\zmean(x) = z(x, \amean)$.
The bed elevation variance $\sigma^2_z(x) = \E[z^2(x)] - \E^2[z(x)]$, which can be approximated by a Taylor series expansion about $\amean$,
\begin{align}
    \sigma_z^2(x) &\approx
    z^2 +
    \left[
    \left(\frac{\partial z}{\partial a}\right)^2
    + z \frac{\partial^2 z}{\partial a^2}
    \right]
    \sigma^2_a
    -
    \left[
    z + \frac{\partial^2 z}{\partial a^2} \frac{\sigma_a^2}{2}
    \right]^2
%
\intertext{where $z = z(x, \amean)$.  Since $\sigma_a$ is small then high order terms can be neglected and the bed elevation variance becomes}
%
    \sigma_z^2(x) &\approx \left( \frac{\partial z(x, \amean)}{\partial a} \sigma_a \right)^2 \label{eqn:z-variance-continuous}
\end{align}
The bump has a Gaussian probability distribution and so high order moments $\mu_3[z(x)], \ldots$ are zero.
Hence, the bed elevation coefficients $z_{i,0}, \ldots, z_{i,P}$ are
\begin{align}
    z_{i,p} = \begin{cases}
    \zmean(x_i) & \text{if $p=0$} \\
    \sigma_z(x_i) & \text{if $p=1$} \\
    0 & \text{otherwise}
    \end{cases}
\end{align}
where $x_i$ is the centre of element $i$ and $z_{i,1}$ is calculated using equations~\eqref{eqn:variance} and \eqref{eqn:z-variance-continuous} with $\mu_2[z(x, \xi)] = \sigma_z^2(x)$.
All polynomial chaos expansions are chosen to have a maximum degree $P = 3$.

The timestep $\Delta t = \SI{0.15}{\second}$ resulting in a maximum Courant number of about $0.8$.  By choosing a fixed timestep, simulations of a given test complete in the same number of timesteps irrespective of the model formulation, and error accumulation due to timestepping errors will be the same across all model formulations.

\todo[inline]{Make data and code available on Zenodo.}

\subsection{\rev{Test 1:} Lake-at-Rest Over an Uncertain Bed}

\rev{Numerical methods that are not well-balanced produce spurious waves in the vicinity of sloping topography, and these spurious waves are particularly evident for slow-moving flows with weak momentum fluxes.
In the limit, the momentum flux is zero and the water is motionless, like the water surface on a lake at rest.
Hence, the lake-at-rest test is ideally suited to verify the well-balanced property, since the analytic solution preserves the resting state forever, and any waves generated by a numerical model are entirely spurious.}

To present a challenging test, a rectangular obstacle is introduced to the right of the uncertain hump given by equation~\eqref{eqn:bed}, so the bed elevation $z$ becomes
\begin{subequations}
\begin{align}
    z(x, \hump) &= \hump \sech^2 \left( \frac{\pi x}{\lambda} \right) + \zrectbump(x) \text{,}
    %
    \intertext{where $\zrectbump$ is the rectangular obstacle:}
    %
    \zrectbump(x) &= \begin{cases}
    \humpmean & \text{if $30 < x \leq 40$,} \\
    0 & \text{otherwise}
    \end{cases}
\end{align}
\end{subequations}

Results of the well-balanced stochastic model are compared with those of a stochastic model having a centred difference approximation of the source term vector that does not exactly balance the numerical flux gradient.
The centred difference model is the same as the well-balanced stochastic model except for two changes.
First, numerical fluxes are calculated using the original, unmodified flow variables:
\begin{align}
	\riemannflux_{i+1/2}^{(n)} = \riemannflux \left(
	\sum_{p=0}^P \flow^-_{i+1/2,p} \pcbasis_p, 
	\sum_{p=0}^P \flow^+_{i+1/2,p} \pcbasis_p
	\right) \label{eqn:flux-centred}
\end{align}
Second, the ensemble average of the source term vector uses a centred difference approximation:
\rev{\begin{align}
    \Ensemble{\source_i \pcbasis_l} =
    \left[ 0, -g \sum_{p=0}^P \sum_{s=0}^P h_{i,p}
    \frac{z_{i+1,s} - z_{i-1,s}}{2 \Delta x}
    \Ensemble{\pcbasis_p \pcbasis_s \pcbasis_l} \right]^\T \label{eqn:source-centred}
\end{align}}
The centred difference model and the well-balanced model are both configured with a Wiener-Hermite basis order $P = 3$.

\begin{figure}
\centering
\begin{subfigure}{\textwidth}
\phantomsubcaption\label{fig:lakeatrest:centred:eta}
\phantomsubcaption\label{fig:lakeatrest:sgm:eta}
\phantomsubcaption\label{fig:lakeatrest:centred:q}
\phantomsubcaption\label{fig:lakeatrest:sgm:q}
\centering
\includegraphics{fig-lakeatrest.pdf}
\end{subfigure}
\caption{Stochastic lake-at-rest solutions at $t = \SI{100}{\second}$.
Mean values are marked by solid lines and shaded regions represent one standard deviation from the mean.}
\label{fig:lakeatrest}
\end{figure}

The simulated time is \SI{100}{\second} corresponding to about 670 timesteps, and the solutions for the centred difference and well-balanced stochastic Galerkin models are shown in figure~\ref{fig:lakeatrest}.
The lack of well-balancing is apparent using the centred difference model: grid-scale standing waves develop at the discontinuities either side of the rectangular hump (figure~\ref{fig:lakeatrest:centred:eta}, \ref{fig:lakeatrest:centred:q}), and a smooth standing wave also develops over the uncertain hump.
These errors persist throughout the simulation.
In contrast, the well-balanced stochastic Galerkin model preserves the initial resting state with discharges accurate to machine precision (figure~\ref{fig:lakeatrest:sgm:q}).
This numerical result confirms that the stochastic Galerkin model is well-balanced in theory and in practice.

The choice of the Wiener-Hermite basis introduces a particular limitation that imposes an upper bound on the basis order $P$, and constrains the minimum water depth that the stochastic Galerkin model can represent.
This limitation arises because the hump amplitude has a Gaussian probability distribution so the tails of the distribution extend to $\pm \infty$, meaning that there is a non-zero probability that the water depth is negative.
The stochastic Galerkin formulation presented here does not accommodate wetting-and-drying processes, and any negative water depth will crash the model.
If the basis order $P$ is increased then the Gauss-Hermite quadrature points in equation~\eqref{eqn:pc-flux} extend further into the tails of the probability distributions, leading to negative water depths being provided as input to the Riemann solver.
Similarly, raising the topography, decreasing the initial water depth, or increasing the topographic uncertainty can all produce negative water depths in the stochastic Galerkin model.
This behaviour has been verified experimentally by varying the model basis order, initial conditions and topography profile.


\subsection{Steady-State Critical Flow Over an Uncertain Bed}

This test is designed to challenge the stochastic Galerkin model at representing highly non-Gaussian distributions of stochastic flow.
The uncertain bed elevation profile and inflow boundary condition are chosen in order to produce a nonlinear response such that the steady-state solution may be subcritical or transcritical depending on the hump amplitude.
Results of the stochastic Galerkin model are compared against a Monte Carlo simulation that serves as the reference solution.

The uncertain bed elevation is the smooth hump given by equation~\eqref{eqn:z-pc-coeffs}.
Subcritical boundary conditions are imposed such that the mean upstream discharge is \SI{1.65}{\meter\squared\per\second} and the mean downstream water elevation is \SI{1.5}{\meter}, with zero uncertainty on both upstream discharge and downstream water elevation.
These boundary conditions are chosen such that the flow is critical over the mean hump amplitude $\amean = \SI{0.6}{\meter}$ at $x = \SI{0}{\meter}$.
Since the hump amplitude is uncertain then the flow regime is also uncertain: if the hump amplitude is less than $\amean$ then the flow remains subcritical; if the hump amplitude is greater than $\amean$ then the flow regime becomes transcritical.

\begin{figure}
    \centering
    \includegraphics{fig-criticalSteadyState-examples.pdf}
    \caption{Well-balanced deterministic solutions using four hump amplitudes, $\amean - \sigma_a = \SI{0.3}{\meter}, \amean = \SI{0.6}{\meter}, \amean+\sigma_a = \SI{0.9}{\meter}$ and $\amean + 2\sigma_a = \SI{1.2}{\meter}$.
    Water elevation is shown with thick lines and the bed elevation profile is shown with thin lines.}
    \label{fig:criticalSteadyState-examples}
\end{figure}

To illustrate this behaviour, figure~\ref{fig:criticalSteadyState-examples} shows four deterministic solutions using different hump amplitudes.
Solutions from the well-balanced deterministic model are obtained at $t = \SI{500}{\second}$ when the water has converged on a steady state, with convergence measured by calculating the $\ell^2$ difference in water height between the current and previous timesteps.
By $t = \SI{500}{\second}$ all four deterministic solutions have converged down to $10^{-4}$.
For a small hump with amplitude $\amean - \sigma_a = \SI{0.3}{\meter}$, the flow remains subcritical.
A linear increase in hump amplitude produces a strongly nonlinear response in the steady-state water profile, as seen in figure~\ref{fig:criticalSteadyState-examples}, 
The upstream boundary condition allows the upstream water elevation to increase nonlinearly, and a transcritical shock develops that increases in amplitude and moves further downstream with larger hump amplitudes.
Downstream of the hump, the mean water elevation is \SI{1.5}{\meter} with zero uncertainty, with this profile having propagated upstream from the imposed downstream boundary.

The stochastic Galerkin model is evaluated by comparing results against a Monte Carlo simulation that serves as a reference solution.
For the Monte Carlo iterations, the topography is generated using a random hump amplitude drawn from the Gaussian distribution given by $(\amean, \sigma_a)$ and so the topography will always be smooth.
If instead the topography was generated using $(\zmean(x), \sigma_z(x))$ then the topography would not be smooth and many more iterations would be needed to sample the stochastic solution space.
The random hump amplitude $a$ is constrained such that $\SI{0}{\meter} \leq a \leq \SI{1.4}{\meter}$ to avoid negative water heights.
2000 Monte Carlo iterations are performed so that the mean and standard deviation of water height are both converged statistically.
Statistical convergence is determined qualitatively, with water height statistics measured at $x = \SI{1.5}{\meter}$ where the variance is large and the probability distribution is highly non-Gaussian.

\subsubsection{Stochastic Galerkin and Monte Carlo solutions}

\begin{figure}
    \centering
    \includegraphics{fig-criticalSteadyState-flow.pdf}
    \caption{Solutions of steady state critical flow over an uncertain hump at $t = \SI{500}{\second}$, comparing stochastic Galerkin and Monte Carlo profiles of mean water elevation $\etamean$ and standard deviation in water elevation $\sigma_\eta$.
    Vertical dotted lines at $x = \SI{-37.5}{\meter}$ and $x = \SI{1.5}{\meter}$ mark the positions of the probability distributions shown in figure~\ref{fig:criticalSteadyState-pdf}.}
    \label{fig:criticalSteadyState-flow}
\end{figure}

In figure~\ref{fig:criticalSteadyState-flow}, spatial profiles of the water elevation statistics are obtained at $t = \SI{500}{\second}$ when the Monte Carlo iterations and stochastic Galerkin solution have converged down to $10^{-4}$.
The stochastic Galerkin model successfully represents the Monte Carlo reference profiles of the mean and standard deviation in water elevation.
Upstream of the hump, the stochastic Galerkin model produces a mean water elevation that is slightly too low, and a standard deviation that is slightly too small compared to the Monte Carlo reference solution.

The mean and standard deviation statistics are useful for summarising the spatial profile of uncertainty, but they are less meaningful for non-Gaussian probability distributions.
Since figure~\ref{fig:criticalSteadyState-examples} demonstrated that the response to an uncertain bed is strongly nonlinear then highly non-Gaussian probability distributions are expected.
Studying the probability distributions is particularly import for flood risk assessments that are concerned with extreme events that occur in the tails of the distributions.

Using the stochastic Galerkin method, the probability density function $f(u)$ for a stochastic variable $u$ can be calculated for a given element $i$ and time level $n$,
\begin{subequations}
\begin{align}
        f(u) = \sum_{k=1}^K \Mag{ \sum_{p=0}^P u_{i,p}^{(n)} \frac{\dee \Phi_p}{\dee \xi}(\randomroot_k)}^{-1} W(\randomroot_k)
%
\intertext{where $\randomroot_k$, $k=1, \ldots, K$ are the real roots of the polynomial}
%
        u - \sum_{p=0}^P u_{i,p}^{(n)} \Phi_p(\xi) = 0
\end{align}
\end{subequations}
which can be calculated numerically for a given value of $u$.

\begin{figure}
    \centering
    \includegraphics{fig-criticalSteadyState-pdf.pdf}
    \caption{Probability distributions of water elevation at (a) $x = \SI{-37.5}{\meter}$ and (b) $x = \SI{1.5}{\meter}$ for steady state critical flow over an uncertain hump at $t = \SI{500}{\second}$.}
    \label{fig:criticalSteadyState-pdf}
\end{figure}

Probability distributions are calculated at $t = \SI{500}{\second}$ at $x = \SI{-37.5}{\meter}$ and $x = \SI{1.5}{\meter}$.
The first point is far upstream of the hump where the water elevation is uncertain and spatially uniform.
The second point is immediately downstream of the hump in the region where transcritical shocks develop if the hump amplitude is sufficiently large.

%This test is designed to challenge stochastic Galerkin methods in representing a bimodal statistical distribution of water elevation

\section{Conclusions and outlook}

This paper proposed a stochastic Galerkin shallow flow model using the surface gradient method to ensure well-balancing.
The well-balancedness of the stochastic model was studied theoretically, and the model was validated using two numerical experiments, both using an idealised topography profile with Gaussian uncertainty.
The first test verified that a lake-at-rest over an uncertain bed was preserved to machine precision.
The second test challenged the stochastic Galerkin model using steady-state critical flow over an uncertain bed.
The test was designed to provoke strongly nonlinear behaviour such that about 50\% of flows were subcritical and the other 50\% develop a transcritical shock.
The resulting probability distributions were discontinuous and highly non-Gaussian.
Without parallelisation, the stochastic Galerkin model was about 100 times faster than a Monte Carlo simulation.
With parallelisation on commodity hardware, the computation time for the stochastic Galerkin model would be on par with a single iteration of the deterministic solver.
Despite its relatively low-order Wiener-Hermite basis, stochastic Galerkin results compared favourably against the Monte Carlo reference solution.

It is likely that a more sophisticated treatment using a multiresolution wavelet decomposition \citep{lemaitre2004a,pettersson2014} or multi-element discretisation of the stochastic dimension \citep{wan-karniadakis2006,li-stinis2015} would further improve the representation of highly non-Gaussian and discontinuous probability distributions.
It is also likely that a decomposition or discretisation of stochastic space would simplify a stochastic reformulation of positivity preserving schemes involving nonlinear operators, so enabling the development of stochastic wetting-and-drying processes.
Such improvements are the subject of future work.
\section*{Acknowledgements}

This work is part of the SEAMLESS-WAVE project (SoftwarE infrAstructure for Multi-purpose fLood modElling at variouS scaleS based on WAVElets) which is funded by the UK Engineering and Physical Sciences Research Council (EPSRC) grant EP/R007349/1.
For information about the SEAMLESS-WAVE project visit \url{https://www.seamlesswave.com}.
\rev{The authors are grateful to the editor and two anonymous reviewers for their helpful comments.}
\appendix
\section{Instructions for running the shallow flow models}

Python 3 software is available for download from Zenodo \citep{shaw-kesserwani2018b} which includes
\begin{enumerate}
    \item a one-dimensional shallow flow model which can operate as a stochastic Galerkin model or a deterministic model
    \item code for running Monte Carlo iterations of the deterministic model
    \item preconfigured lake-at-rest and steady-state critical flow test cases
\end{enumerate}
First ensure that Python 3, NumPy and SciPy are installed.
Then, install the shallow flow model:
\begin{verbatim}
pip3 install --user --editable swepc.python
\end{verbatim}

\subsection{Running the Stochastic Galerkin Model}
To run the stochastic Galerkin model:
\begin{Verbatim}[commandchars=\\\{\}]
swepc --degree \textit{<P>} \textit{<testCase>} \textit{<discretisation>} --output-dir \textit{<directory>}
\end{Verbatim}
where \Verb[commandchars=\\\{\}]+\textit{<testCase>}+ is either \Verb+lakeAtRest+ or \Verb+criticalSteadyState+,
\Verb[commandchars=\\\{\}]+\textit{<discretisation>}+ is either \Verb+wellBalancedH+ for the well-balanced surface gradient method, or \Verb+centredDifferenceH+ for the centred difference method given by equations~\eqref{eqn:flux-centred} and \eqref{eqn:source-centred}.
If \Verb[commandchars=\\\{\}]+--degree \textit{<P>}+ is omitted then the default basis order $P = 3$ is used.
If \Verb[commandchars=\\\{\}]+--degree 0+ is specified then the model operates as a deterministic model.
On running the model, the following text files are written to the specified output directory:
\begin{description}
\item[\Verb+coefficients.dat+]{Stochastic Galerkin expansion coefficients for topography $z$, water height $h$ and discharge $q$}
\item[\Verb+statistics.dat+]{Mean, standard deviation, skew and kurtosis calculated using equations~\eqref{eqn:moment}, \eqref{eqn:mean} and \eqref{eqn:variance}}
\item[\Verb+derived-statistics.dat+]{Statistics for the free-surface elevation, a derived quantity calculated from equation~\eqref{eqn:h-eta-z}}
\end{description}
Each file has one line per element with explanatory header rows prefixed by \Verb+#+.

\subsection{Calculating Probability Density Functions}
A probability density function for any variable can be calculated from the expansion coefficients for a given element:
\begin{Verbatim}[commandchars=\\\{\}]
sed -n \textit{<line>}p coefficients.dat
   | swepdf --min \textit{<value>} --max \textit{<value>} \textit{<variable>} > pdf.dat
\end{Verbatim}
where \Verb[commandchars=\\\{\}]+\textit{<line>}+ is the line number in \Verb+coefficients.dat+ corresponding to the chosen element, and \Verb[commandchars=\\\{\}]+\textit{<variable>}+ is \Verb+z+ (topography), \Verb+water+ (water height), \Verb+q+ (discharge) or \Verb+derived-eta+ (free-surface elevation).
The probability density function is calculated between the \Verb+--min+ and \Verb+--max+ values and written to \Verb+pdf.dat+. 

\subsection{Running Monte Carlo Simulations}
To run a Monte Carlo simulation:
\begin{Verbatim}[commandchars=\\\{\}]
swepc --monte-carlo --mc-iterations \textit{<value>}
  criticalSteadyState \textit{<discretisation>} --output-dir \textit{<directory>}
\end{Verbatim}
which writes the following text files:
\begin{description}
\item[\Verb+statistics.dat+]{Mean, standard deviation, skew and kurtosis for topography, water height and discharge}
\item[\Verb+derived-statistics.dat+]{Mean and standard deviation for free-surface elevation}
\cprotect\item[\Verb[commandchars=\\\{\}]+sample\textit{<n>}.dat+]{Deterministic model data with one file per element (numbered $0, \ldots, M-1$), each having one line per Monte Carlo iteration}
\end{description}
\bibliography{references}

\end{document}
