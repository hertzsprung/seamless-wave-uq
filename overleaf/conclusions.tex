\section{Conclusions and outlook}

This paper proposed a stochastic Galerkin shallow flow model using the surface gradient method to ensure well-balancing.
The well-balancedness of the stochastic model was studied theoretically and verified in a series of idealised numerical tests, with a prescribed uncertainty that characterises topographic or bathymetric measurement errors.
These tests demonstrated that the stochastic Galerkin model preserves discrete balance between flux and topography gradients to machine precision, and accurately reproduces steady flow over highly irregular and uncertain topography.
Despite its relatively low-order Wiener-Hermite basis, the stochastic Galerkin model was able to simulate strongly nonlinear behaviour, producing discontinuous and highly non-Gaussian probability distributions that compared favourably against a Monte Carlo reference solution.

It is likely that a more sophisticated treatment using a multiresolution wavelet decomposition \citep{lemaitre2004a,pettersson2014} or multi-element discretisation of the stochastic dimension \citep{wan-karniadakis2006,li-stinis2015} would further improve the representation of highly non-Gaussian and discontinuous probability distributions.
It is also likely that a decomposition or discretisation of stochastic space would simplify a stochastic reformulation of positivity preserving schemes involving nonlinear operators, so enabling the development of stochastic wetting-and-drying processes that are essential for flood inundation modelling.
Such improvements are the subject of future work.

The one-dimensional stochastic Galerkin model presented here is ideally suited to river routing applications:
using as few as two expansion coefficients per element, the stochastic model can account for river bed measurement errors to produce a probabilistic flow simulation that is accurate and computationally efficient.
Without parallelisation, the stochastic Galerkin model is at least 100 times faster than a Monte Carlo simulation.
With parallelisation on commodity hardware, the computation time for the stochastic Galerkin model would be on par with a single iteration of the deterministic solver.

%The first test verified that a lake-at-rest over an uncertain bed was preserved to machine precision.
%The second test challenged the stochastic Galerkin model using steady-state critical flow over an uncertain bed.
%The test was designed to provoke strongly nonlinear behaviour such that about 50\% of flows were subcritical and the other 50\% develop a transcritical shock.
%The resulting probability distributions were discontinuous and highly non-Gaussian.
%\rev{A third test verified that the stochastic Galerkin model accurately simulates steady flow over a highly irregular and uncertain bed profile.}



