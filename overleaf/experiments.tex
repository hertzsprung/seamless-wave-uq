\section{Idealised numerical tests}

Three numerical tests are performed to validate the well-balanced stochastic Galerkin model.
The first test simulates a lake-at-rest over idealised, uncertain topography to verify that the stochastic Galerkin model preserves well-balancing numerically.
The second test simulates flow over a hump with an uncertain elevation.
This test is designed to challenge the stochastic Galerkin model at representing discontinuous, non-Gaussian probability distributions by generating a steady-state solution that may be subcritical or transcritical depending on the hump elevation.
The third test verifies the robustness of the stochastic Galerkin model for a steady flow over a highly irregular and uncertain bed that is more representative of real-world river hydraulics.
\rev{The configuration of these three tests is summarised in table~\ref{tab:results}.}

\begin{table}
\centering
\begin{tabular}{llll}
\toprule
    & 1: Lake-at-rest & 2: Steady-state critical flow & 3: Flow over an irregular bed \\
%    Test case & Mesh & $\Delta x$ & $\Delta t$ & End time & Topography & Models compared & Key findings \\
\midrule
    Purpose & Verify well-balancedness & Nonlinear, non-Gaussian flow & Realistic topography \\
    Mesh & $[ \SI{-50}{\meter}, \SI{50}{\meter} ]$ & $[ \SI{-50}{\meter}, \SI{50}{\meter} ]$ & $[ \SI{0}{\meter}, \SI{1500}{\meter} ]$ \\
    $\Delta x$ & \SI{1}{\meter} & \SI{1}{\meter} & \SI{7.5}{\meter} \\
    $\Delta t$ & \SI{0.15}{\second} & \SI{0.15}{\second} & \SI{0.5}{\second} \\
    End time & \SI{100}{\second} & \SI{500}{\second} & \SI{100000}{\second} \\
    Topography & Equation~\eqref{eqn:lakeAtRest:z} & Equation~\eqref{eqn:bed} & \citet{goutal-maurel1997} \\
%    Models & Centred difference; Well-balanced & Well-balanced; Monte Carlo & Well-balanced only \\
\bottomrule
\end{tabular}
\caption{Summary of the three idealised numerical test cases.}
\label{tab:results}
\end{table}

\subsection{Specification of Tests 1 and 2}
For the first two tests, the 1D domain is [\SI{-50}{\meter}, \SI{50}{\meter}], tessellated by $M = 100$ elements with no overlaps or gaps such that the mesh spacing is $\Delta x = \SI{1}{\meter}$.
The timestep is $\Delta t = \SI{0.15}{\second}$ resulting in a maximum Courant number of about $0.8$.
By choosing a fixed timestep, simulations of a given test complete in the same number of timesteps irrespective of the model configuration, and error accumulation due to timestepping errors will be the same across all models.

Both tests include a topographic hump centred at $x = \SI{0}{\meter}$ with a region of Gaussian uncertainty.
Following a similar approach to \citet{ge2008}, there are two representations for the same uncertain topography.
The first representation enables smooth topography profiles to be randomly generated in Monte Carlo iterations.
The topography $z$ is defined as
\begin{align}
z(x, \hump) = \hump \sech^2 \left( \frac{\pi x}{\lambda} \right) \label{eqn:bed}
\end{align}
where the hump amplitude $\hump$ is a random variable with mean $\humpmean = \SI{0.6}{\meter}$ and standard deviation $\sigma_\hump = \SI{0.3}{\meter}$, and the half-width is $\lambda = \SI{10}{\meter}$.
This topography profile is seen in figure~\ref{fig:criticalSteadyState-flow}.
The second representation is used by the stochastic Galerkin model, with topography represented by expansion coefficients $z_{i,0}, \ldots, z_{i,P}$.
To be able to calculate the topography expansion coefficients, equation~\eqref{eqn:bed} must be expressed in terms of mean topography $\zmean(x)$ and topographic variance $\sigma_z^2(x)$ without involving the random variable $\hump$.
The mean topography $\zmean(x)$ is simply
\begin{align}
    \zmean(x) = z(x, \humpmean)
    \label{eqn:bed-mean}
\end{align}
The topographic variance is
\begin{align}
    \sigma^2_z(x) = \E[z^2(x, \hump)] - \E^2[z(x, \hump)]
    \label{eqn:bed-variance}
\end{align}
Equation~\eqref{eqn:bed-variance} can be rewritten using Taylor series expansions of the two terms $\E\left[z^2(x, \hump)\right]$ and $\E^2\left[z(x, \hump)\right]$.
To illustrate the approach, a Taylor series expansion of $\E\left[ z(x, \hump) \right]$ about $\humpmean$ is 
\begin{align}
    \E\left[ z(x, \hump) \right] &= \E\left[ z(x, \humpmean + (\hump - \humpmean)) \right] \nonumber \\
    &= \E\left[ z + \frac{\partial z}{\partial \hump} (\hump - \humpmean) + \frac{1}{2} \frac{\partial^2 z}{\partial \hump^2} \left(\hump - \humpmean\right)^2 + \mathcal{O}(\hump^3) \right]
    \label{eqn:mean-taylor}
%
\intertext{where $z$ is shorthand for $z(x, \humpmean)$ and $\mathcal{O}(\hump^3)$ is the error term involving high-order derivatives $\partial^m z/\partial \hump^m$ with $m \geq 3$.
Since $\E\left[ \hump - \humpmean \right] = 0$ and $\E\left[ \left(\hump-\humpmean\right)^2\right] = \sigma_\hump^2$ then equation~\eqref{eqn:mean-taylor} simplifies to}
%
    \E\left[ z(x, \hump) \right] &= z + \frac{1}{2}\frac{\partial^2 z}{\partial \hump^2} \sigma_\hump^2 + \mathcal{O}(\hump^3)
\end{align}
Applying this approach to equation~\eqref{eqn:bed-variance} gives:
\begin{align}
    \sigma_z^2(x) &=
    z^2 +
    \left[
    \left(\frac{\partial z}{\partial \hump}\right)^2
    + z \frac{\partial^2 z}{\partial \hump^2}
    \right]
    \sigma^2_\hump
    -
    \left[
    z + \frac{1}{2} \frac{\partial^2 z}{\partial \hump^2} \sigma_\hump^2
    \right]^2 + \mathcal{O}(\hump^3) \label{eqn:z-taylor}
%
\intertext{For the topographic profile given by equation~\eqref{eqn:bed}, it holds that $\partial^m z/\partial \hump^m = 0$ where $m \geq 2$, so the Taylor series approximation introduces no spurious oscillations in stochastic space.
The topographic variance in equation~\eqref{eqn:z-taylor} then simplifies to}
%
    \sigma_z^2(x) &= \left( \frac{\partial z(x, \humpmean)}{\partial \hump} \sigma_\hump \right)^2 \label{eqn:z-variance-continuous}
\end{align}
Equipped with analytic expressions for the mean topography $\zmean(x)$ (equation~\ref{eqn:bed-mean}) and topographic variance $\sigma_z^2(x)$ (equation~\ref{eqn:z-variance-continuous}), now the topography expansion coefficients $z_{i,0}, \ldots, z_{i,P}$ can be calculated.
Since the topographic bump has a Gaussian probability distribution with $\mu_1[z(x)] = \zmean(x)$, $\mu_2[z(x)] = \sigma_z^2(x)$ and high-order moments $\mu_m[z(x)] = 0$ for $m \geq 3$ then, using equations~\eqref{eqn:moment}, \eqref{eqn:mean} and \eqref{eqn:variance}, the topography expansion coefficients are
\begin{align}
    z_{i,p} = \begin{cases}
    \zmean(x_i) & \text{if $p=0$} \\
    \sigma_z(x_i) & \text{if $p=1$} \\
    0 & \text{otherwise}
    \end{cases}
    \label{eqn:z-pc-coeffs}
\end{align}
where values are calculated at the centre point $x_i$ for all elements $i=1,\ldots, M$.

%and $z_{i,1}$ is calculated using equations~\eqref{eqn:z-variance-continuous} and \eqref{eqn:variance} with $\mu_2[z(x, \xi)] = \sigma_z^2(x)$.

The initial water depth expansion coefficients $h_{i,0}^{(0)}, \ldots, h_{i,P}^{(0)}$ can be calculated in terms of free-surface elevation and topography using equation~\eqref{eqn:h-eta-z}.
For both tests, the initial, spatially-uniform mean free-surface elevation is \SI{1.5}{\meter} with no initial uncertainty such that $\eta_{i,0}^{(0)} = \SI{1.5}{\meter}$ and $\eta_{i,p}^{(0)} = 0$ with $p = 1, \ldots, P$ and $i = 1, \ldots, M$.

\subsection{\rev{Test 1:} Lake-at-Rest Over an Uncertain Bed}

\rev{Numerical methods that are not well-balanced produce spurious waves in the vicinity of sloping topography, and these spurious waves are particularly evident for slow-moving flows with weak momentum fluxes.
In the limit, the momentum flux is zero and the water is motionless, like the water surface on a lake at rest.
Hence, the lake-at-rest test is ideally suited to verify the well-balanced property, since the analytic solution preserves the resting state forever, and any waves generated by a numerical model are entirely spurious.}

To present a challenging test, a rectangular obstacle is introduced to the right of the uncertain hump given by equation~\eqref{eqn:bed}, so the bed elevation $z$ becomes
\begin{subequations}
\begin{align}
    z(x, \hump) &= \hump \sech^2 \left( \frac{\pi x}{\lambda} \right) + \zrectbump(x) \text{,}
    %
    \intertext{where $\zrectbump$ is the rectangular obstacle:}
    %
    \zrectbump(x) &= \begin{cases}
    \humpmean & \text{if $30 < x \leq 40$,} \\
    0 & \text{otherwise}
    \end{cases}
\end{align}
\end{subequations}

Results of the well-balanced stochastic model are compared with those of a stochastic model having a centred difference approximation of the source term vector that does not exactly balance the numerical flux gradient.
The centred difference model is the same as the well-balanced stochastic model except for two changes.
First, numerical fluxes are calculated using the original, unmodified flow variables:
\begin{align}
	\riemannflux_{i+1/2}^{(n)} = \riemannflux \left(
	\sum_{p=0}^P \flow^-_{i+1/2,p} \pcbasis_p, 
	\sum_{p=0}^P \flow^+_{i+1/2,p} \pcbasis_p
	\right) \label{eqn:flux-centred}
\end{align}
Second, the ensemble average of the source term vector uses a centred difference approximation:
\rev{\begin{align}
    \Ensemble{\source_i \pcbasis_l} =
    \left[ 0, -g \sum_{p=0}^P \sum_{s=0}^P h_{i,p}
    \frac{z_{i+1,s} - z_{i-1,s}}{2 \Delta x}
    \Ensemble{\pcbasis_p \pcbasis_s \pcbasis_l} \right]^\T \label{eqn:source-centred}
\end{align}}
The centred difference model and the well-balanced model are both configured with a Wiener-Hermite basis order $P = 3$.

\begin{figure}
\centering
\begin{subfigure}{\textwidth}
\phantomsubcaption\label{fig:lakeatrest:centred:eta}
\phantomsubcaption\label{fig:lakeatrest:sgm:eta}
\phantomsubcaption\label{fig:lakeatrest:centred:q}
\phantomsubcaption\label{fig:lakeatrest:sgm:q}
\centering
\includegraphics{fig-lakeatrest.pdf}
\end{subfigure}
\caption{Stochastic lake-at-rest solutions at $t = \SI{100}{\second}$.
Mean values are marked by solid lines and shaded regions represent one standard deviation from the mean.}
\label{fig:lakeatrest}
\end{figure}

The simulated time is \SI{100}{\second} corresponding to about 670 timesteps, and the solutions for the centred difference and well-balanced stochastic Galerkin models are shown in figure~\ref{fig:lakeatrest}.
The lack of well-balancing is apparent using the centred difference model: grid-scale standing waves develop at the discontinuities either side of the rectangular hump (figure~\ref{fig:lakeatrest:centred:eta}, \ref{fig:lakeatrest:centred:q}), and a smooth standing wave also develops over the uncertain hump.
These errors persist throughout the simulation.
In contrast, the well-balanced stochastic Galerkin model preserves the initial resting state with discharges accurate to machine precision (figure~\ref{fig:lakeatrest:sgm:q}).
This numerical result confirms that the stochastic Galerkin model is well-balanced in theory and in practice.

The choice of the Wiener-Hermite basis introduces a particular limitation that imposes an upper bound on the basis order $P$, and constrains the minimum water depth that the stochastic Galerkin model can represent.
This limitation arises because the hump amplitude has a Gaussian probability distribution so the tails of the distribution extend to $\pm \infty$, meaning that there is a non-zero probability that the water depth is negative.
The stochastic Galerkin formulation presented here does not accommodate wetting-and-drying processes, and any negative water depth will crash the model.
If the basis order $P$ is increased then the Gauss-Hermite quadrature points in equation~\eqref{eqn:pc-flux} extend further into the tails of the probability distributions, leading to negative water depths being provided as input to the Riemann solver.
Similarly, raising the topography, decreasing the initial water depth, or increasing the topographic uncertainty can all produce negative water depths in the stochastic Galerkin model.
This behaviour has been verified experimentally by varying the model basis order, initial conditions and topography profile.


\subsection{Steady-State Critical Flow Over an Uncertain Bed}

This test is designed to challenge the stochastic Galerkin model at representing highly non-Gaussian distributions of stochastic flow.
The uncertain bed elevation profile and inflow boundary condition are chosen in order to produce a nonlinear response such that the steady-state solution may be subcritical or transcritical depending on the hump amplitude.
Results of the stochastic Galerkin model are compared against a Monte Carlo simulation that serves as the reference solution.

The uncertain bed elevation is the smooth hump given by equation~\eqref{eqn:z-pc-coeffs}.
Subcritical boundary conditions are imposed such that the mean upstream discharge is \SI{1.65}{\meter\squared\per\second} and the mean downstream water elevation is \SI{1.5}{\meter}, with zero uncertainty on both upstream discharge and downstream water elevation.
These boundary conditions are chosen such that the flow is critical over the mean hump amplitude $\amean = \SI{0.6}{\meter}$ at $x = \SI{0}{\meter}$.
Since the hump amplitude is uncertain then the flow regime is also uncertain: if the hump amplitude is less than $\amean$ then the flow remains subcritical; if the hump amplitude is greater than $\amean$ then the flow regime becomes transcritical.

\begin{figure}
    \centering
    \includegraphics{fig-criticalSteadyState-examples.pdf}
    \caption{Well-balanced deterministic solutions using four hump amplitudes, $\amean - \sigma_a = \SI{0.3}{\meter}, \amean = \SI{0.6}{\meter}, \amean+\sigma_a = \SI{0.9}{\meter}$ and $\amean + 2\sigma_a = \SI{1.2}{\meter}$.
    Water elevation is shown with thick lines and the bed elevation profile is shown with thin lines.}
    \label{fig:criticalSteadyState-examples}
\end{figure}

To illustrate this behaviour, figure~\ref{fig:criticalSteadyState-examples} shows four deterministic solutions using different hump amplitudes.
Solutions from the well-balanced deterministic model are obtained at $t = \SI{500}{\second}$ when the water has converged on a steady state, with convergence measured by calculating the $\ell^2$ difference in water height between the current and previous timesteps.
By $t = \SI{500}{\second}$ all four deterministic solutions have converged down to $10^{-4}$.
For a small hump with amplitude $\amean - \sigma_a = \SI{0.3}{\meter}$, the flow remains subcritical.
A linear increase in hump amplitude produces a strongly nonlinear response in the steady-state water profile, as seen in figure~\ref{fig:criticalSteadyState-examples}, 
The upstream boundary condition allows the upstream water elevation to increase nonlinearly, and a transcritical shock develops that increases in amplitude and moves further downstream with larger hump amplitudes.
Downstream of the hump, the mean water elevation is \SI{1.5}{\meter} with zero uncertainty, with this profile having propagated upstream from the imposed downstream boundary.

The stochastic Galerkin model is evaluated by comparing results against a Monte Carlo simulation that serves as a reference solution.
For the Monte Carlo iterations, the topography is generated using a random hump amplitude drawn from the Gaussian distribution given by $(\amean, \sigma_a)$ and so the topography will always be smooth.
If instead the topography was generated using $(\zmean(x), \sigma_z(x))$ then the topography would not be smooth and many more iterations would be needed to sample the stochastic solution space.
The random hump amplitude $a$ is constrained such that $\SI{0}{\meter} \leq a \leq \SI{1.4}{\meter}$ to avoid negative water heights.
2000 Monte Carlo iterations are performed so that the mean and standard deviation of water height are both converged statistically.
Statistical convergence is determined qualitatively, with water height statistics measured at $x = \SI{1.5}{\meter}$ where the variance is large and the probability distribution is highly non-Gaussian.

\subsubsection{Stochastic Galerkin and Monte Carlo solutions}

\begin{figure}
    \centering
    \includegraphics{fig-criticalSteadyState-flow.pdf}
    \caption{Solutions of steady state critical flow over an uncertain hump at $t = \SI{500}{\second}$, comparing stochastic Galerkin and Monte Carlo profiles of mean water elevation $\etamean$ and standard deviation in water elevation $\sigma_\eta$.
    Vertical dotted lines at $x = \SI{-37.5}{\meter}$ and $x = \SI{1.5}{\meter}$ mark the positions of the probability distributions shown in figure~\ref{fig:criticalSteadyState-pdf}.}
    \label{fig:criticalSteadyState-flow}
\end{figure}

In figure~\ref{fig:criticalSteadyState-flow}, spatial profiles of the water elevation statistics are obtained at $t = \SI{500}{\second}$ when the Monte Carlo iterations and stochastic Galerkin solution have converged down to $10^{-4}$.
The stochastic Galerkin model successfully represents the Monte Carlo reference profiles of the mean and standard deviation in water elevation.
Upstream of the hump, the stochastic Galerkin model produces a mean water elevation that is slightly too low, and a standard deviation that is slightly too small compared to the Monte Carlo reference solution.

The mean and standard deviation statistics are useful for summarising the spatial profile of uncertainty, but they are less meaningful for non-Gaussian probability distributions.
Since figure~\ref{fig:criticalSteadyState-examples} demonstrated that the response to an uncertain bed is strongly nonlinear then highly non-Gaussian probability distributions are expected.
Studying the probability distributions is particularly import for flood risk assessments that are concerned with extreme events that occur in the tails of the distributions.

Using the stochastic Galerkin method, the probability density function $f(u)$ for a stochastic variable $u$ can be calculated for a given element $i$ and time level $n$,
\begin{subequations}
\begin{align}
        f(u) = \sum_{k=1}^K \Mag{ \sum_{p=0}^P u_{i,p}^{(n)} \frac{\dee \Phi_p}{\dee \xi}(\randomroot_k)}^{-1} W(\randomroot_k)
%
\intertext{where $\randomroot_k$, $k=1, \ldots, K$ are the real roots of the polynomial}
%
        u - \sum_{p=0}^P u_{i,p}^{(n)} \Phi_p(\xi) = 0
\end{align}
\end{subequations}
which can be calculated numerically for a given value of $u$.

\begin{figure}
    \centering
    \includegraphics{fig-criticalSteadyState-pdf.pdf}
    \caption{Probability distributions of water elevation at (a) $x = \SI{-37.5}{\meter}$ and (b) $x = \SI{1.5}{\meter}$ for steady state critical flow over an uncertain hump at $t = \SI{500}{\second}$.}
    \label{fig:criticalSteadyState-pdf}
\end{figure}

Probability distributions are calculated at $t = \SI{500}{\second}$ at $x = \SI{-37.5}{\meter}$ and $x = \SI{1.5}{\meter}$.
The first point is far upstream of the hump where the water elevation is uncertain and spatially uniform.
The second point is immediately downstream of the hump in the region where transcritical shocks develop if the hump amplitude is sufficiently large.

%This test is designed to challenge stochastic Galerkin methods in representing a bimodal statistical distribution of water elevation

\subsection{Test 3: Steady-State Slow Flow Over an Irregular Bed}
While the previous tests are restricted to idealised topography profiles, this final test simulates slow flow over a highly irregular bed that is more representative of real-world river hydraulics.
The bed is defined with a range of uncertainty that characterises bathymetric measurement errors.
Stochastic Galerkin model results are verified against the analytic solution.

For this purpose, following \citet{tseng2004}, the 1D domain is $[\SI{0}{\meter}, \SI{1500}{\meter}]$ with $M = 200$ elements such that the mesh spacing is $\Delta x = \SI{7.5}{\meter}$.
The mean topography has an irregular profile as specified by \citet{goutal-maurel1997} and is shown in figure~\ref{fig:tsengSteadyState-flow:elevation}.
The standard deviation of topography is $\sigma_z(x) = \SI{0.5}{\meter}$ across the entire domain, which is chosen since it provides an acceptable margin of error for local-scale flood modelling applications \citep{schumann-bates2018}.
The initial free-surface elevation is \SI{15}{\meter} with no uncertainty.
Subcritical boundary conditions are imposed such that the mean upstream discharge per unit-width is \SI{0.75}{\meter\squared\per\second} and the mean downstream water depth is \SI{15}{\meter}, with no uncertainty on either boundary condition.
Transmissive boundary conditions are used for the upstream water depth and downstream discharge.
The test uses a timestep of $\Delta t = \SI{0.5}{\second}$ and terminates at $t = \SI{100000}{\second}$ when the water has converged on a steady state down to a convergence error of \SI{e-8}{\meter} as measured by equation~\eqref{eqn:convergence}.

\begin{figure}
    \centering
    \begin{subfigure}{\textwidth}
    \centering
    \phantomsubcaption\label{fig:tsengSteadyState-flow:elevation}
    \phantomsubcaption\label{fig:tsengSteadyState-flow:u}
    \includegraphics{fig-tsengSteadyState-flow.pdf}
    \end{subfigure}
    \caption{Steady state solution over an irregular and uncertain bed.
    (a) Bed elevation and free-surface elevation.
    Mean values are marked with solid lines, with shading indicating plus or minus one standard deviation of uncertainty.
    (b) Spatial profile of depth-average velocity.
    The mean velocity is marked with a solid line and the standard deviation of velocity is negligible.
    In both panels, the analytic mean solution is marked with a dashed yellow line.}
    \label{fig:tsengSteadyState-flow}
\end{figure}

The stochastic Galerkin model is configured with a basis order $P=3$, and the resulting steady-state solution is shown in figure~\ref{fig:tsengSteadyState-flow}.
Since the flow is very slow, the free-surface elevation is uniform.
The free-surface elevation has the same standard deviation as the topography, which is due to the water depth at the downstream boundary being fixed at \SI{15}{\meter} with no uncertainty.
As a result, the water depth profile is certain, and the uncertainty in the free-surface elevation is determined entirely by the topography.
The corresponding depth-averaged velocity profile is shown in figure~\ref{fig:tsengSteadyState-flow:u}.
The stochastic Galerkin model calculates a velocity profile that is in excellent agreement with the analytic solution.
The predicted velocity profile is certain with negligible standard deviation.
This is expected since the water depth profile is unaffected by the prescribed topographic uncertainty.
These results verify the robustness of the stochastic Galerkin model for flow over a highly-irregular, uncertain bed.