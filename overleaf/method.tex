\section{Deterministic and stochastic shallow flow models}

In this section, a certain deterministic numerical solver of the one dimensional (1D) shallow water equations is outlined in the framework of a finite volume Godunov-type method.
The selected deterministic solver relies on the surface gradient method \citep{zhou2001} to both ensure a well-balanced topography integration and extendibility of the well-balanced property into the stochastic Galerkin case. 
Accordingly, a stochastic Galerkin reformulation is devised that is theoretically well-balanced with uncertain topography under a lake-at-rest hypothesis.

The mathematical model of the shallow water equations represent mass and momentum
conservation principals, and is used in the following conservative form when solving it within a finite volume Godunov-type framework \citep{toro-garcianavarro2007}:
\begin{align}
\frac{\partial \flow(x, t)}{\partial t} + \frac{\partial \flux(\flow(x, t))}{\partial x} = \source(\flow(x, t), z(x)) \label{eqn:swe}
\end{align}
where $\flow = \left[ h, q \right]^\T$ is the flow vector including water height $h$ ($\mathrm{L}$) and unit-width discharge $q = h\velocity$ ($\mathrm{L}^2/\mathrm{T}$) in which $\velocity$ represents the depth-averaged velocity ($\mathrm{L}/\mathrm{T}$), $\flux = \left[ q,  q^2/h + gh^2/2 \right]^\T$ is the flux vector in which $g$ represents the gravitational constant and $\source = \left[ 0, -gh \: \dee z / \dee x \right]^\T$ is the source term vector in which the gradient of the topography $z(x)$ is involved.

\subsection{Deterministic model}

On a uniform 1D mesh with $M$ elements each of size $\Delta x$, the first-order finite volume method leads to the following discrete element-wise formulation of equation~\eqref{eqn:swe}:
\begin{align}
    \flow_i^{(n+1)} = \flow_i^{(n)} - \Delta t
    \left(
    \frac{\riemannflux_{i+1/2}^{(n)} - \riemannflux_{i-1/2}^{(n)}}{\Delta x}
    - \source_i^{(n)} \right) \label{eqn:swe-discrete}
\end{align}
in which $\flow_i^{(n)} = \left[ h_i^{(n)}, q_i^{(n)} \right]^\T$ is a piecewise-constant discretisation of the flow vector at element $i$ and time level $(n)$, and $\riemannflux_{i+1/2}^{(n)}$ is a numerical flux function for linking nonlinear discontinuities associated with the flow vector data at interface $i+1/2$ located between element $i$ and $i + 1$.
Namely, $\riemannflux_{i+1/2}^{(n)} = \riemannflux(\flow_{i+1/2}^-, \flow_{i+1/2}^+)$ where $\flow_{i+1/2}^-$ is the limit of the solution from the side of element $i$, and $\flow_{i+1/2}^+$ is the limit of the solution from the side of element $i+1$.
Within the scope of this work involving a first-order accurate solver, these limits become $\flow_{i+1/2}^- = \flow_i$ and $\flow_{i+1/2}^+ = \flow_{i+1}$ that are used in a numerical flux function based on the approximate Riemann solver of Roe \citep{roe-pike1984}.

The surface gradient method essentially reconstructs an averaged topography at interface $i+1/2$ that is shared by both elements $i$ and $i+1$, as $\zmodified_{i+1/2} = (z_i + z_{i+1})/2$.
From the reconstructed topography $\zmodified_{i+1/2}$, consistent flow variable limits are accordingly reconstructed based on the actual free-surface elevation data, i.e. $\eta_{i+1/2}^- = h_i^{(n)} + z_i$ and $\eta_{i+1/2}^+ = h_{i+1}^{(n)} + z_{i+1}$, and velocity data, i.e. $\velocity_{i+1/2}^- = q_i^{(n)}/h_i^{(n)}$ and $\velocity_{i+1/2}^+ = q_{i+1}^{(n)} / h_{i+1}^{(n)}$, as: $\hKmodified_{i+1/2} = \eta_{i+1/2}^K - \zmodified_{i+1/2}$ and $\qKmodified_{i+1/2} = \hKmodified_{i+1/2} \velocity_{i+1/2}^K$ (where $K = + \text{ or } -$).
For clarity of presentation, the time level denoted by superscript $(n)$ is omitted from all reconstructed variables.
These reconstructions form new Riemann states $\flowKmodified_{i+1/2} = \left[ \hKmodified_{i+1/2}, \qKmodified_{i+1/2} \right]^\T$ for use to evaluate $\riemannflux_{i+1/2}^{(n)}$.
By analogy, new Riemann limits $\flowKmodified_{i-1/2} = \left[ \hKmodified_{i-1/2}, \qKmodified_{i-1/2} \right]^\T$ at $i - 1/2$ are produced for use to evaluate $\riemannflux_{i-1/2}^{(n)}$.
From the reconstructed limits, a well-balanced discretisation of the source term vector can be produced:
\begin{align}
	\source_i^{(n)} = \left[ 0, -g
	\left( \frac{h^{+,\star}_{i-1/2} + h^{-,\star}_{i+1/2}}{2} \right)
	\left( \frac{z^\star_{i+1/2} - z^\star_{i-1/2}}{\Delta x} \right)
	\right]^\T
	\label{eqn:source}
\end{align}
The well-balanced deterministic model presented in equations~\eqref{eqn:swe-discrete} and \eqref{eqn:source} is used for Monte Carlo simulations, and the deterministic model is also the starting point for a stochastic Galerkin reformulation.

\subsection{Fundamental properties of the Polynomial Chaos basis}

Before presenting the stochastic Galerkin reformulation, it is necessary to consider a single random variable $X(\theta)$ that maps from the random event $\theta$ to an arbitrary probability distribution with finite variance.
This random variable can be approximated by a Wiener-Hermite Polynomial Chaos expansion \citep{xiu-karniadakis2002}.
The expansion is based on a standard Gaussian random variable $\xi(\theta) \in [-\infty, +\infty]$ having zero mean and unit variance.
The random variable of interest, $X(\theta)$, is then approximated as
\begin{align}
X(\theta) \approx \sum_{p=0}^P X_p \pcbasis_p(\xi(\theta))
\end{align}
where $\vect{X} = \left[ X_0, \ldots, X_P \right]^\T$ are the expansion coefficients and $\pcbasisvect = \left[ \pcbasis_0, \ldots, \pcbasis_P \right]^\T$ is the probabilists' Hermite polynomial basis having basis function $\pcbasis_p$ of degree $p$,
\begin{align}
    \pcbasis_p(\xi) = \left( -1 \right)^p \exp \left(\frac{\xi^2}{2}\right)
    \frac{\dee^p}{\dee \xi^p} \exp \left(- \frac{\xi^2}{2} \right)
\end{align}
where $\pcbasis_0 = 1, \pcbasis_1 = \xi, \pcbasis_2 = \xi^2 - 1, \pcbasis_3 = \xi^3 - 3\xi$ and so on.
As the basis order $P$ is increased, the Wiener-Hermite Polynomial Chaos approximation converges on the true random variable $X(\theta)$ \citep{xiu-karniadakis2002}.

\subsubsection*{Basis orthogonality and commutativity}
The Wiener-Hermite basis $\pcbasisvect$ is orthogonal such that
\begin{align}
	\Ensemble{\pcbasis_p \pcbasis_s} = \Ensemble{\pcbasis_p^2} \delta_{ps}
\end{align}
where $\Ensemble{\cdot}$ is the ensemble average operator and $\delta_{ps}$ is the Kronecker delta that is equal to one when $p = s$ and zero otherwise.
The ensemble average operator is defined as the weighted integral over the standard Gaussian random variable $\xi$:
\begin{align}
	\Ensemble{\alpha(\xi)} = \int_{-\infty}^\infty \alpha(\xi) W(\xi) \diff \xi \label{eqn:ensemble-average}
\end{align}
where $\alpha(\xi)$ is an expression involving any combination of random variables or basis functions, and the weighting function $W(\xi)$ is the standard Gaussian probability density function
\begin{align}
	W(\xi) = \frac{1}{\sqrt{2\pi}} \exp \left(-\frac{\xi^2}{2}\right)
\end{align}
This weighting function ensures that, when $\alpha$ is independent of $\xi$, the ensemble average $\Ensemble{\alpha} = \alpha$.
Finally, the ensemble average of a product of basis functions is commutative such that
\begin{align}
    \Ensemble{\pcbasis_p \pcbasis_s} = \Ensemble{\pcbasis_s \pcbasis_p}
    \label{eqn:commutative}
\end{align}
The commutative property is needed later when verifying the well-balanced property with uncertain topography.

\subsubsection*{Stochastic Galerkin projection of a random variable}
Given the random variable $X(\theta) = \sum_{p=0}^P X_p \pcbasis_p(\xi(\theta))$, its Galerkin projection onto a basis function $\pcbasis_l$ with $l = 0, \ldots, P$ is achieved using the ensemble average operator such that, due to orthogonality,
\begin{align}
	\Ensemble{X(\theta) \pcbasis_l} = X_l \Ensemble{\pcbasis_l^2} \label{eqn:orthogonal}
\end{align}
where $X_l$ is the $l$\textsuperscript{th} order expansion coefficient.
Also note that the Galerkin projection of a basis function $\pcbasis_l$ and two random variables, $X(\theta)$ and $Y(\theta)$, is distributive:
\begin{align}
	\Ensemble{\left(X(\theta) + Y(\theta)\right) \pcbasis_l}
	=
	\Ensemble{X(\theta) \pcbasis_l} + \Ensemble{Y(\theta) \pcbasis_l} \label{eqn:distributive}
\end{align}

\subsubsection*{Mean, variance and high-order moments}
The mean, variance and high-order moments can be calculated for $X(\theta)$.
The $m$\textsuperscript{th} moment $\mu_m[X]$ is defined as
\begin{align}
\mu_m[X] = \int_{-\infty}^\infty \left(X - \beta \right)^m W(\xi) \diff \xi
    =
    \Ensemble{\left( X - \beta \right)^m} \label{eqn:moment}
\end{align}
where $\beta = 0$ when $m = 1$ and $\beta = \mu_1[X]$ for higher-order moments.
Therefore, the mean $\mu_1[X] = \Ensemble{X} = \sum_{p=0}^P X_p \Ensemble{\pcbasis_p}$.
Since $\Ensemble{\pcbasis_0} = 1$ and $\Ensemble{\pcbasis_p} = 0$ for $p > 0$ then
\begin{align}
\mu_1[X] = X_0
\label{eqn:mean}
\end{align}
The shorthand notation for the mean of $X$ is $\mean{X}$, also known as the expected value, $\E\left[X\right]$.

The variance $\mu_2[X]$ can be derived using the fact that $\E\left[ \left( X - \E[X] \right)^2 \right] = \E[X^2] - \E^2[X]$, hence $\mu_2[X] = \left(\sum_{p=0}^P X_p^2 \Ensemble{\pcbasis_p^2}\right) - X_0^2 \Ensemble{\pcbasis_0}^2$.
Since $\Ensemble{\pcbasis_0}^2 = \Ensemble{\pcbasis_0^2}$ then
\begin{align}
    \mu_2[X] &= \sum_{p=1}^P X^2_p \Ensemble{\pcbasis_p^2} \label{eqn:variance}
\end{align}
The shorthand notation for the variance of $X$ is $\sigma^2_X$ and the standard deviation of $X$ is $\sigma_X$.

\subsubsection*{Reconstructing the probability density function}
The probability density function $f_X(x)$ of a random variable $X$ is,
\begin{subequations}
\begin{align}
        f_X(x) = \sum_{j=1}^J \Mag{ \sum_{p=0}^P X_p \frac{\dee \Phi_p}{\dee \xi}(\randomroot_j)}^{-1} W(\randomroot_j)
%
\intertext{where $\randomroot_j$, $j=1, \ldots, J$ are the real roots of the polynomial}
%
        x - \sum_{p=0}^P X_p \pcbasis_p(\xi) = 0
\end{align}\label{eqn:pdf}%
\end{subequations}
which can be calculated numerically for a specific outcome $x$.
Hence, the probability density function is computed by evaluating equation~\eqref{eqn:pdf} for a range of outcomes.

\subsection{Stochastic Galerkin reformulation of the deterministic model}
%In the deterministic 1D shallow water equations (equation~\ref{eqn:swe}), the flow vector is $\flow(x, t)$ and the topography is $z(x)$.
The solution of the stochastic 1D shallow water equations is now random because it depends on uncertain initial conditions, uncertain boundary conditions and uncertain topography.
Hence, the stochastic 1D shallow water equations depend not only upon space $x$ and time $t$, but additionally upon the random event $\theta$.
The stochastic flow vector $\flow(x, t, \theta)$ becomes a general stochastic process having arbitrary probability distributions that vary in space and time.
Similarly, the stochastic topography $z(x, \theta)$ has arbitrary probability distributions that vary in space.

The stochastic Galerkin reformulation of the deterministic model involves three steps to (i) replace the deterministic variables, $\flow_i^{(n)}$ and $z_i$, with random variables $\flow_i^{(n)}(\theta)$ and $z_i(\theta)$, (ii) rewrite the deterministic formulation using these random variables, and (iii) make a stochastic Galerkin projection onto the Wiener-Hermite basis.

\subsubsection*{Replacing deterministic variables with random variables}

For all elements $i=1, \ldots, M$ across all time levels, every deterministic flow variable $\flow_i^{(n)} = \left[h_i^{(n)}, q_i^{(n)}\right]^\T$ and deterministic topography variable $z_i$ becomes a random variable approximated by a Wiener-Hermite Polynomial Chaos expansion:
\begin{align}
\flow_i^{(n)}(\theta) \approx \sum _{p=0}^P \flow_{i,p}^{(n)} \pcbasis_p(\xi(\theta))
    \:\text{,}\quad
z_i(\theta) \approx \sum_{p=0}^P z_{i,p} \pcbasis_p(\xi(\theta))
\label{eqn:pc-expansion}%
\end{align}
where $\flow_{i,p}^{(n)} = \left[ h_{i,p}^{(n)}, q_{i,p}^{(n)} \right]^\T$ and $z_{i,p}$ are the $p$\textsuperscript{th} order expansion coefficients over element $i$ at time level $n$.

The reconstructed topography and reconstructed limits become functions of random variables.
The reconstructed topography at interface $i+1/2$ becomes
\begin{align}
	\sum_{p=0}^P z^\star_{i+1/2,p} \pcbasis_p
	=
	\frac{1}{2}
	\left(
	\sum_{p=0}^P z_{i,p} \pcbasis_p
	+
	\sum_{p=0}^P z_{i+1,p} \pcbasis_p
	\right)
\end{align}
and so $z^\star_{i+1/2,p} = (z_{i,p} + z_{i+1,p})/2$ due to basis orthogonality.
The reconstructed limits $\flow^{K,\star}_{i+1/2,p} = \left[ h^{K,\star}_{i+1/2,p}, q^{K,\star}_{i+1/2,p} \right]^\T$ (where $K = + \text{ or } -$) are calculated in a similar fashion.

Random variables that are functions of other random variables can be calculated in the same way.
In particular, water height can be expressed as a function of free-surface elevation and topography such that, due to basis orthogonality,
\begin{align}
h_{i,p}^{(n)} = \eta_{i,p}^{(n)} - z_{i,p}
\label{eqn:h-eta-z}
\end{align}
which is used later for specifying initial conditions.

\subsubsection*{Rewriting the deterministic formulation using random variables}

The deterministic finite volume formulation given by equation~\eqref{eqn:swe-discrete} is rewritten in terms of the random variables in equation~\eqref{eqn:pc-expansion}.
As a result, the numerical fluxes $\riemannflux_{i+1/2}^{(n)}$ and $\riemannflux_{i-1/2}^{(n)}$ and source term vector $\source_i^{(n)}$ become functions of random variables.
The numerical flux $\riemannflux_{i+1/2}^{(n)}$ becomes
\begin{align}
	\riemannflux_{i+1/2}^{(n)} = \riemannflux \left(
	\sum_{p=0}^P \flow^{-,\star}_{i+1/2,p} \pcbasis_p, 
	\sum_{p=0}^P \flow^{+,\star}_{i+1/2,p} \pcbasis_p
	\right)
\end{align}
and similarly for $\riemannflux_{i-1/2}^{(n)}$.
The source term vector $\source_i^{(n)}$ in equation~\eqref{eqn:source} becomes
\begin{align}
	\source_i^{(n)} = \left[ 0, -\frac{g}{\Delta x}
	\left\{
	\sum_{p=0}^P \frac{h^{+,\star}_{i-1/2,p} + h^{-,\star}_{i+1/2,p}}{2} \pcbasis_p \right\}
\left\{ \sum_{s=0}^P \left( z^\star_{i+1/2,s} - z^\star_{i-1/2,s} \right) \pcbasis_s \right\}
	\right]^\T
	\label{eqn:random-source}
\end{align}
Equation~\eqref{eqn:random-source} involves the product of two expressions, each delimited by braces.
Since both expressions include Wiener-Hermite expansions then different indices, $p$ and $s$, are needed for the expansion coefficients in each expression.

\subsubsection*{Stochastic Galerkin projection}

Due to the orthogonal property (equation~\ref{eqn:orthogonal}) and distributive property (equation~\ref{eqn:distributive}) of the Wiener-Hermite basis, a Galerkin projection of equation~\eqref{eqn:swe-discrete} onto the basis functions $\pcbasis_l, l = 0, \ldots, P$ produces $P+1$ decoupled equations:
\begin{align}
    \flow_{i,l}^{(n+1)} = \flow_{i,l}^{(n)}
    - \frac{\Delta t}{\Ensemble{\pcbasis_l^2}}
    \left(
    \frac{
    \Ensemble{\riemannflux_{i+1/2}^{(n)} \pcbasis_l}
    -
    \Ensemble{\riemannflux_{i-1/2}^{(n)} \pcbasis_l}
    }{\Delta x}
    - \Ensemble{\source_i^{(n)} \pcbasis_l}
    \right) \label{eqn:swe-pc}
\end{align}
Equation~\eqref{eqn:swe-pc} involves ensemble averages of numerical fluxes, $\Ensemble{\riemannflux_{i+1/2}^{(n)} \pcbasis_l}$ and $\Ensemble{\riemannflux_{i-1/2}^{(n)} \pcbasis_l}$, and an ensemble average of the source term vector, $\Ensemble{\source_i^{(n)} \pcbasis_l}$.
There is no straightforward method for calculating an ensemble average of the numerical flux because it is nonlinear.
Instead, the integral in equation~\eqref{eqn:ensemble-average} is approximated by Gauss-Hermite quadrature,
\begin{align}
    \Ensemble{\riemannflux_{i+1/2}^{(n)} \pcbasis_l}
    \approx
    \sum_{j=1}^{P+1} w_j
    \riemannflux\left(
	\sum_{p=0}^P \flow_{i+1/2,p}^{-,\star} \pcbasis_p(\xi_j),
	\sum_{p=0}^P \flow_{i+1/2,p}^{+,\star} \pcbasis_p(\xi_j)
	\right)
    \pcbasis_l(\xi_j) W(\xi_j) \label{eqn:pc-flux}
\end{align}
where $w_j$ are the quadrature weights and $\xi_j$ are the quadrature points.
The ensemble average $\Ensemble{\riemannflux_{i-1/2}^{(n)} \pcbasis_l}$ is calculated in the same way.

Unlike the nonlinear numerical flux, the ensemble average of the source term vector $\Ensemble{\source_i^{(n)} \pcbasis_l}$ is linear and can be derived directly from equation~\eqref{eqn:random-source}:
\begin{align}
\Ensemble{\source_i^{(n)} \pcbasis_l} &= \left[ 0,
    - \frac{g}{\Delta x}
    \sum_{p=0}^P \sum_{s=0}^P
\left(\frac{h^{+,\star}_{i-1/2,p} + h^{-,\star}_{i+1/2,p}}{2}\right)
\left( z^\star_{i+1/2,s} - z^\star_{i-1/2,s} \right)
    \Ensemble{\pcbasis_p \pcbasis_s \pcbasis_l}
    \right]^\T
\label{eqn:pc-source}
\end{align}
The ensemble averages $\Ensemble{\pcbasis_p \pcbasis_s \pcbasis_l}$ in equation~\eqref{eqn:pc-source} and $\Ensemble{\pcbasis_l^2}$ in equation~\eqref{eqn:swe-pc} can be calculated analytically or exactly by Gauss-Hermite quadrature.
Since these calculations do not depend on the solution then they can be precomputed once and stored.
