\section{Deterministic and stochastic Galerkin shallow water model formulations}

In this section, the deterministic one-dimensional shallow water equations are discretised using a shock-capturing first-order finite volume method.
To ensure well-balancing in the deterministic model, the bed slope source term is discretised using the surface gradient method \citep{zhou2001}.
Using the stochastic Galerkin method, a stochastic dimension is introduced into the deterministic model, and a polynomial chaos expansion is made to produce the corresponding stochastic model.

The deterministic one-dimensional shallow water equation is
\begin{align}
    \frac{\partial \flow}{\partial t} + \frac{\partial \flux(\flow)}{\partial x} = \source \label{eqn:swe}
\end{align}
where $\flow$ is the flow, $\flux$ is the flux and $\source$ is the bed slope source term.
The flow $\flow = \left[ h, q \right]^\T$ with water height $h$, discharge per unit width $q = hu$ and velocity $u$.
The flux $\flux = \left[ q,  q^2/h + gh^2/2 \right]^\T$ where the gravitational acceleration $g = \SI{9.80665}{\meter\per\second}$.  The source term $\source = \left[ 0, -gh \: \dee z / \dee x \right]^\T$ where $z(x)$ is the bed elevation profile.

\subsection{Deterministic Shallow Water Model Formulation}
Equation~\eqref{eqn:swe} is discretised on a one-dimensional domain with $M$ elements and uniform mesh spacing $\Delta x$.
Applying the first-order finite volume method and forward Euler time-stepping yields the evolution equation
\begin{align}
    \flow_i^{(n+1)} = \flow_i^{(n)} - \frac{\Delta t}{\Delta x}
    \left( \riemannflux_{i+1/2}^{(n)} - \riemannflux_{i-1/2}^{(n)}
    - \Delta x \source_i^{(n)} \right) \label{eqn:swe-discrete}
\end{align}
which evolves the flow on element $i = 1, \ldots, M$ from time level $(n)$ to time level $(n+1)$ with a time-step $\Delta t$.  
The numerical flux $\riemannflux_{i+1/2}$ is calculated at interface $i+1/2$ using a Roe approximate Riemann solver with entropy correction\todo{citation}.

To ensure well-balancing in the deterministic model, the bed slope source term $\source_i$ is discretised using the surface gradient method \citep{zhou2001},
\begin{align}
\source_i = \left[ 0, -g \hmodified_i \frac{\Delta \zmodified_i}{\Delta x} \right]^\T \label{eqn:source-h}
\end{align}
where the modified water height $\hmodified_i$ is
\begin{align}
    \hmodified_i = \frac{h_{i-1/2}^+ + h_{i+1/2}^-}{2} \label{eqn:hbar}
\end{align}
and the modified water heights at the interfaces of element $i$ are $h_{i-1/2}^+ = \eta_{i-1/2}^+ - \zmodified_{i-1/2}$ and $h_{i+1/2}^- = \eta_{i+1/2}^- - \zmodified_{i+1/2}$ where the water elevation $\eta = h + z$.
Since the first-order finite volume discretisation uses a piecewise-constant basis then $\eta_{i-1/2}^+ = \eta_{i+1/2}^- = h_i + z_i$.
The change in bed elevation $\Delta \zmodified_i$ over element $i$ is
\begin{align}
    \Delta \zmodified_i = \zmodified_{i+1/2} - \zmodified_{i-1/2} \label{eqn:delta-z}
\end{align}
where the averaged bed elevation $\zmodified_{i+1/2}$ at interface $i+1/2$ is
\begin{align}
\zmodified_{i+1/2} = \frac{z_i + z_{i+1}}{2} \label{eqn:z-interface}
\end{align}
The numerical flux $\riemannflux_{i+1/2}$ is calculated based on the averaged bed elevation $\zmodified_{i+1/2}$ such that $\riemannflux_{i+1/2} = \riemannflux(\flow_{i+1/2}^-, \flow_{i+1/2}^+)$ where $\flow_{i+1/2}^- = [h_{i+1/2}^-, q_{i+1/2}^-]^\T$ and $\flow_{i-1/2}^+ = [h_{i+1/2}^+, q_{i+1/2}^+]^\T$.
The discharges $q_{i-1/2}^+$ and $q_{i+1/2}^-$ are calculated using the velocity $u_i = q_i/h_i$ such that $q_{i-1/2}^+ = h_{i-1/2}^+ \, u_i$ and $q_{i+1/2}^- = h_{i+1/2}^- \, u_i$.

\subsection{Stochastic Galerkin Shallow Water Model Formulation}
The stochastic Galerkin method introduces a stochastic dimension with coordinate $\xi$ such that the flow $\flow = \flow(x, t, \xi)$ and the bed elevation $z = z(x, \xi)$.
The flow and bed elevation fields in the stochastic dimension are represented by a polynomial chaos expansion,
\begin{align}
    \flow(x, t, \xi) = \sum _{p=0}^P \flow_p(x, t) \pcbasis_p(\xi) \quad\text{,}\quad
    z(x, \xi) = \sum _{p=0}^P z_p(x) \pcbasis_p(\xi)
\end{align}\label{eqn:pc-expansion}%
where $\pcbasisvect = \left[ \pcbasis_0, \ldots, \pcbasis_P \right]$ is the probabilists' Hermite polynomial basis having term $\pcbasis_p$ with degree $p$,
\begin{align}
    \pcbasis_p(\xi) = \left( -1 \right)^p \exp \left(\frac{\xi^2}{2}\right)
    \frac{\dee^p}{\dee \xi^p} \exp \left(- \frac{\xi^2}{2} \right)
\end{align}
where $\pcbasis_0 = 1, \pcbasis_1 = \xi, \pcbasis_2 = \xi^2 - 1, \pcbasis_3 = \xi^3 - 3\xi$ and so on.
The ensemble average $\Ensemble{\pcbasis_p \pcbasis_\palt}$ is 
\begin{align}
    \Ensemble{\pcbasis_p \pcbasis_\palt} = \int_{-\infty}^\infty \pcbasis_p \pcbasis_\palt W \diff \xi
\end{align}
where the Gaussian probability density function $W$ is
\begin{align}
    W = \frac{1}{\sqrt{2\pi}} \exp \left(-\frac{\xi^2}{2}\right)
\end{align}
Since the polynomial chaos basis is orthogonal then $\Ensemble{\pcbasis_p \pcbasis_\palt} = \Ensemble{\pcbasis_p^2} \delta_{p\palt}$ where $\delta_{p\palt}$ is the Kronecker delta.

The stochastic Galerkin shallow water model is formulated by making a polynomial chaos expansion of equation~\eqref{eqn:swe-discrete} then making a Galerkin projection onto the bases $\pcbasis_l, l = 0, \ldots, P$ to produce $P + 1$ decoupled equations,
\begin{align}
    \flow_{i,l}^{(n+1)} = \flow_{i,l}^{(n)}
    - \frac{\Delta t}{\Delta x \Ensemble{\pcbasis_l}^2}
    \left(
    \Ensemble{\riemannflux_{i+1/2}^{(n)} \pcbasis_l}
    -
    \Ensemble{\riemannflux_{i-1/2}^{(n)} \pcbasis_l}
    - \Delta x \Ensemble{\source_i^{(n)} \pcbasis_l}
    \right) \label{eqn:swe-pc}
\end{align}
where $\flow_{i,l}^{(n)}$ are the flow coefficients at time level $n$ over element $i$ in the physical dimension, $l$ in the stochastic dimension.
The stochastic evolution equation~\eqref{eqn:swe-pc} involves ensemble averages of numerical fluxes and an ensemble average of the bed slope source term.
Since the ensemble average of the numerical flux $\Ensemble{\riemannflux_{i+1/2} \pcbasis_l}$ is nonlinear, it is approximated by Gauss-Hermite quadrature,
\begin{align}
    \Ensemble{\riemannflux_{i+1/2} \pcbasis_l}
    \approx
    \sum_{j=1}^{P+1} w_j
    \riemannflux( \flow_{i+1/2}^-(\xi_j), \flow_{i+1/2}^+(\xi_j) )
    \pcbasis_l(\xi_j) W(\xi_j) \label{eqn:pc-flux}
\end{align}
where $w_j$ are the quadrature weights and $\xi_j$ are the quadrature points in the stochastic dimension.
%The numerical flux $\riemannflux( \flow_{i+1/2}^-(\xi_j), \flow_{i+1/2}^+(\xi_j) )$ is calculated using the flow evaluated at the quadrature points $\xi_j$ such that
%\begin{align}
%    \flow_{i+1/2}^-(\xi_j) = \sum_{p=0}^P \flow_{i+1/2,p}^- \pcbasis_p(\xi_j)
%\end{align}
%and similarly for $\flow_{i+1/2}^+$.
%\todo[inline]{
%This is a minor point but took me a little time to work out.
%It is straightforward to calculate the modified water heights at the interfaces since addition of polynomial chaos expansions is distributive: $\sum h_p \pcbasis_p + \sum z_p \pcbasis_p = \sum (h_p + z_p) \pcbasis_p$.
%However, the modified discharges at the interfaces involve division ($u = q/h$) and multiplication $q^\pm = u*h^\pm$ which are not distributive operations.
%Hence, the modified discharge must be evaluated for given quadrature points $\xi_j$, and cannot be precomputed using the coefficients alone.}

The stochastic bed slope source term $\Ensemble{\source_i \pcbasis_l}$ in equation~\eqref{eqn:swe-pc} is derived from the deterministic bed slope source term (equation~\ref{eqn:source-h}) such that
\begin{align}
    \Ensemble{\source_i \pcbasis_l} &= \left[ 0,
    - \frac{g}{\Delta x}
    \sum_{p=0}^P \sum_{\palt=0}^P
    \hmodified_{i,p} \Delta \zmodified_{i,\palt}
    \Ensemble{\pcbasis_p \pcbasis_\palt \pcbasis_l}
    \right]^\T
\label{eqn:pc-source-h}
\end{align}
The ensemble averages of polynomial chaos bases in equation~\eqref{eqn:pc-source-h}, and in the denominator of equation~\eqref{eqn:swe-pc}, can be calculated analytically or by Gauss-Hermite quadrature.

\todo[inline]{Mention that only continuous functions are amenable to a stochastic Galerkin formulation.
Show how a well-balancing method that uses min or max cannot be decoupled.
Very quick sketch, say we need to calculate $\zmodified_i = \max(z_{i-1}, z_{i+1})$.
Making a polynomial chaos expansion we have
\begin{align}
    \sum_{p=0}^P \zmodified_{i,p} \pcbasis_p
    =
    \max\left(\sum_{p=0}^P z_{i-1,p} \pcbasis_p, 
    \sum_{p=0}^P z_{i+1,p} \pcbasis_p \right)
%
\intertext{A stochastic Galerkin projection gives}
%
\Ensemble{\zmodified_{i,l} \pcbasis_l^2}
=
    \Ensemble{\max\left(\sum_{p=0}^P z_{i-1,p} \pcbasis_p, 
    \sum_{p=0}^P z_{i+1,p} \pcbasis_p \right)
    \pcbasis_l}
\end{align}
and now we're stuck!  I wonder if we can integrate this by quadrature or something...}