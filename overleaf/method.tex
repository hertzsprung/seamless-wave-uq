\section{Deterministic and stochastic shallow water model formulations}

In this section, the deterministic one-dimensional shallow water equations with a topographic source term are formulated using either water height or water elevation, resulting in two deterministic formulations called $h$-form and $\eta$-form.
The two deterministic formulations are discretised using a shock-capturing first-order finite volume method.
The bed slope source terms are discretised using the surface gradient method \citep{zhou2001,liang-borthwick2009}.
Using the stochastic Galerkin method, a stochastic dimension is introduced into the deterministic $h$-form and $\eta$-form models, and a polynomial chaos expansion is made to produce the two corresponding stochastic models.

The deterministic one-dimensional shallow water equation is
\begin{align}
    \frac{\partial \flow}{\partial t} + \frac{\partial \flux(\flow)}{\partial x} = \source \label{eqn:swe}
\end{align}
where $\flow$ is the flow, $\flux$ is the flux and $\source$ is the bed slope source term.
The flow can be expressed in terms of water height $h$, or water elevation $\eta = h + z$ where $z$ is the bed elevation, resulting in two formulations called $h$-form and $\eta$-form.
For the $h$-form model the flow $\flow$ is
\begin{subequations}
\begin{align}
    \flow &= \left[ h, q \right]^\T
%
\intertext{with discharge per unit width $q = hu$ and velocity $u$.  The flux $\flux$ is}
%
    \flux &= \left[ q,  \frac{q^2}{h} + \frac{gh^2}{2} \right]^\T
%
\intertext{where the gravitational acceleration $g = \SI{9.80665}{\meter\per\second}$.  The source term $\source$ is}
%
    \source &= \left[ 0, -gh \frac{\dee z}{\dee x} \right]^\T
\end{align}
\end{subequations}
For the $\eta$-form model the flow, flux and source term are \citep{liang-borthwick2009}
\begin{subequations}
\begin{align}
    \flow &= \left[ \eta, q \right]^\T \\
    \flux &= \left[ q, \frac{q^2}{\eta - z} +
            \frac{g}{2}\left(\eta^2 - 2\eta z \right) \right]^\T \\
    \source &= \left[ 0, -g \eta \frac{\dee z}{\dee x} \right]^\T
\end{align}
\end{subequations}

\subsection{Discretisation of the Deterministic $h$-form and $\eta$-form Finite Volume Shallow Water Models}
Equation~\eqref{eqn:swe} is discretised on a one-dimensional domain with $M$ elements and uniform mesh spacing $\Delta x$.
Applying the first-order finite volume method and forward Euler time-stepping yields the evolution equation
\begin{align}
    \flow_i^{(n+1)} = \flow_i^{(n)} - \frac{\Delta t}{\Delta x}
    \left( \riemannflux_{i+1/2}^{(n)} - \riemannflux_{i-1/2}^{(n)}
    - \Delta x \source_i^{(n)} \right) \label{eqn:swe-discrete}
\end{align}
which evolves the flow on element $i = 1, \ldots, M$ from time level $(n)$ to time level $(n+1)$ with a time-step $\Delta t$.  
The numerical flux $\riemannflux_{i+1/2}$ is calculated at the interface $i+1/2$  using a Roe approximate Riemann solver with entropy correction\todo{citation}.

To ensure well-balancing in the deterministic models, the bed slope source term $\source_i$ is discretised using the surface gradient method.
For the $\eta$-form model, the source term $\source_i$ is \citep{liang-borthwick2009}
\begin{align}
    \source_i = \left[ 0, -g \eta_i \frac{\Delta z_i}{\Delta x} \right]^\T \label{eqn:source-eta}
\end{align}
where $\Delta z_i$ is calculated based on the averaged bed elevation at interfaces.  The averaged bed elevation $z_{i+1/2}^\pm$ at interface $i+1/2$ is
\begin{align}
z_{i+1/2}^\pm = \frac{z_i + z_{i+1}}{2} \label{eqn:z-interface}
\end{align}
and the change in bed elevation $\Delta z_i$ over element $i$ is
\begin{align}
    \Delta z_i = z_{i+1/2}^\pm - z_{i-1/2}^\pm \label{eqn:delta-z}
\end{align}
For the $h$-form model, the source term is \citep{zhou2001}
\begin{align}
\source_i = \left[ 0, -g \bar{h}_i \frac{\Delta z_i}{\Delta x} \right]^\T \label{eqn:source-h}
\end{align}
where the modified water height $\bar{h}_i$ is
\begin{align}
    \bar{h}_i = \frac{h_{i-1/2}^+ + h_{i+1/2}^-}{2} \label{eqn:hbar}
\end{align}
The modified water heights at the interfaces of element $i$ are
\begin{align}
    h_{i-1/2}^+ &= \eta_{i-1/2}^+ - z_{i-1/2}^\pm \\
    h_{i+1/2}^- &= \eta_{i+1/2}^- - z_{i+1/2}^\pm
\end{align}
For both $h$-form and $\eta$-form models, the numerical flux $\riemannflux_{i+1/2}$ is calculated using the averaged bed elevation $z_{i+1/2}^\pm$ such that
\begin{align}
    \riemannflux_{i+1/2} = \riemannflux(\flow_{i+1/2}^-, \flow_{i+1/2}^+)
    = \riemannflux\left(
    \begin{bmatrix}
        h_{i+1/2}^- \\
        q_{i+1/2}^-
    \end{bmatrix}
    ,
     \begin{bmatrix}
        h_{i+1/2}^+ \\
        q_{i+1/2}^+
    \end{bmatrix}
    \right)
\end{align}
Since the first-order finite volume discretisation uses a piecewise-constant basis then $\eta_{i-1/2}^+ = \eta_{i+1/2}^- = \eta_i = h_i + z_i$ and $q_{i-1/2}^+ = q_{i+1/2}^- = q_i$.

\subsection{Stochastic Galerkin Method}
The stochastic Galerkin method introduces a stochastic dimension with coordinate $\xi$ such that the flow $\flow = \flow(x, t, \xi)$ and the bed elevation $z = z(x, \xi)$.
The flow and bed elevation fields in the stochastic dimension are represented by a polynomial chaos expansion,
\begin{subequations}
\begin{align}
    \flow(x, t, \xi) &= \sum _{p=0}^P \flow_p(x, t) \pcbasis_p(\xi) \\
    z(x, \xi) &= \sum _{p=0}^P z_p(x) \pcbasis_p(\xi)
\end{align}\label{eqn:pc-expansion}%
\end{subequations}
where $\pcbasisvect = \left[ \pcbasis_0, \ldots, \pcbasis_P \right]$ is the probabilists' Hermite polynomial basis having term $\pcbasis_p$ with degree $p$,
\begin{align}
    \pcbasis_p(\xi) = \left( -1 \right)^p \exp \left(\frac{\xi^2}{2}\right)
    \frac{\dee^p}{\dee \xi^p} \exp \left(- \frac{\xi^2}{2} \right)
\end{align}
where $\pcbasis_0 = 1, \pcbasis_1 = \xi, \pcbasis_2 = \xi^2 - 1, \pcbasis_3 = \xi^3 - 3\xi$ and so on.
The ensemble average $\Ensemble{\pcbasis_p \pcbasis_q}$ is 
\begin{align}
    \Ensemble{\pcbasis_p \pcbasis_q} = \int_{-\infty}^\infty \pcbasis_p \pcbasis_q W \diff \xi
\end{align}
where the Gaussian probability density function $W$ is
\begin{align}
    W = \frac{1}{\sqrt{2\pi}} \exp \left(-\frac{\xi^2}{2}\right)
\end{align}
Since the polynomial chaos basis is orthogonal then $\Ensemble{\pcbasis_p \pcbasis_q} = \Ensemble{\pcbasis_p^2} \delta_{pq}$ where $\delta_{pq}$ is the Kronecker delta.

The $m$-th moment $\mu_m[u]$ of a flow variable $u = h(x, t, \xi)$ or $u = q(x, y, \xi)$ or bed elevation $u = z(x, t, \xi)$ is
\begin{align}
    \mu_m [u] = \int_{-\infty}^\infty \left(u - c\right)^m W \diff \xi
\end{align}
where $c = 0$ when $m = 1$ and $c = \mu_0[u]$ for higher moments.
Hence, the mean $\mu_1[u] = u_0$ and the variance $\mu_2[u]$ is
%Hence, the mean flow $\mu_1[\flow(x, t, \xi)] = \flow_0(x, t)$ and the mean bed elevation $\mu_1[z(x, \xi)] = z_0(x)$.
\begin{align}
    \mu_2[u] &= \sum_{p=1}^P u^2_p \Ensemble{\pcbasis_p^2} \label{eqn:variance}
\end{align}
where the zeroth-order term with $p = 0$ is omitted because the variance is centred about the mean.

\subsection{Stochastic Galerkin Shallow Water Model Formulations}
A stochastic Galerkin shallow water model is formulated by making a polynomial chaos expansion of equation~\eqref{eqn:swe-discrete} then making a Galerkin projection onto the bases $\pcbasis_l, l = 0, \ldots, P$ to produce $P + 1$ decoupled equations,
\begin{align}
    \flow_{i,l}^{(n+1)} = \flow_{i,l}^{(n)}
    - \frac{\Delta t}{\Delta x \Ensemble{\pcbasis_l}^2}
    \left(
    \Ensemble{\riemannflux_{i+1/2}^{(n)} \pcbasis_l}
    -
    \Ensemble{\riemannflux_{i-1/2}^{(n)} \pcbasis_l}
    - \Delta x \Ensemble{\source_i^{(n)} \pcbasis_l}
    \right) \label{eqn:swe-pc}
\end{align}
where $\flow_{i,l}^{(n)}$ are the flow coefficients at time level $n$ over element $i$ in the physical dimension, $l$ in the stochastic dimension.
The stochastic evolution equation~\eqref{eqn:swe-pc} involves ensemble averages of numerical fluxes and an ensemble average of the bed slope source term.
The ensemble average of the numerical flux $\Ensemble{\riemannflux_{i+1/2} \pcbasis_l}$ can be approximated by Gauss-Hermite quadrature,
\begin{align}
    \Ensemble{\riemannflux_{i+1/2} \pcbasis_l}
    \approx
    \sum_{j=1}^{P+1} w_j
    \riemannflux( \flow_{i+1/2}^-(\xi_j), \flow_{i+1/2}^+(\xi_j) )
    \pcbasis_l(\xi_j) W(\xi_j)
\end{align}
where $w_j$ are the quadrature weights and $\xi_j$ are the quadrature points in the stochastic dimension.
The numerical flux $\riemannflux( \flow_{i+1/2}^-(\xi_j), \flow_{i+1/2}^+(\xi_j) )$ is calculated using the flow evaluated at the quadrature points $\xi_j$ such that
\begin{align}
    \flow_{i+1/2}^-(\xi_j) = \sum_{p=0}^P \flow_{i+1/2,p}^- \pcbasis_p(\xi_j)
\end{align}
and similarly for $\flow_{i+1/2}^+$.

The stochastic source term for the $\eta$-form model is derived from equation~\eqref{eqn:source-eta},
\begin{align}
    \Ensemble{\source_i \pcbasis_l} = \left[ 0,
    -\frac{g}{\Delta x} \sum_{p=0}^P \sum_{q=0}^P \eta_{i,p}
    \Delta z_{i,q}
    \Ensemble{\pcbasis_p \pcbasis_q \pcbasis_l}
    \right]^\T \label{eqn:pc-source-eta}
\end{align}
where $\Delta z_{i,q} = z_{i+1/2,q}^\pm - z_{i-1/2,q}^\pm$ and $z_{i+1/2,q}^\pm = (z_{i,q}+z_{i+1,q})/2$.
The double summation in equation~\eqref{eqn:pc-source-eta} is a result of the tensor product that arises from the polynomial chaos expansions of water elevation $\eta$ and bed elevation $z$.

The stochastic source term for the $h$-form model is derived from equation~\eqref{eqn:source-h}.  The derivation is more complicated because $\bar{h}_i$ and $\Delta z_i$ are both functions of bed elevation $z$ .
By using the definitions of $\bar{h}_i$ (equation~\ref{eqn:hbar}) and $\Delta z_i$ (equations~\ref{eqn:z-interface} and \ref{eqn:delta-z}), the deterministic source term $\source_i$ (equation~\ref{eqn:source-h}) can be rewritten as
\begin{align}
\source_i = \left[ 0, -\frac{g}{\Delta x} \left\{
         \left(h_i + \frac{z_i}{2} \right) \Delta z_i
        - \frac{1}{8} \left( z_{i+1}^2 - z_{i-1}^2 \right)
        \right\} \right]^\T
\end{align}
Hence the stochastic source term for the $h$-form model is
\begin{align}
    \Ensemble{\source_i \pcbasis_l} &= \left[ 0,
    - \frac{g}{\Delta x} \left\{
    \sum_{p=0}^P \sum_{q=0}^P
    \left[
    \left( h_{i,p} + \frac{z_{i,p}}{2} \right) \Delta z_{i,q} 
    - \frac{1}{8}
    \left( z_{i+1,p} z_{i+1,q} - z_{i-1,p} z_{i-1,q} \right)
    \right]
    \Ensemble{\pcbasis_p \pcbasis_q \pcbasis_l}
    \right\}
    \right]^\T
\label{eqn:pc-source-h}
\end{align}
The ensemble averages of polynomial chaos bases in equations~\eqref{eqn:pc-source-eta}, \eqref{eqn:pc-source-h}, and in the denominator of equation~\eqref{eqn:swe-pc}, can be calculated analytically or by Gauss-Hermite quadrature.

\todo[inline]{Mention that only continuous functions are amenable to a stochastic Galerkin formulation.
Show how a well-balancing method that uses min or max cannot be decoupled.}
